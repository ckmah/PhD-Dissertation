\begin{dissertationepilogue}

    % Fix a bug in the numbering of the epilogue chapters?
    \setcounter{chapter}{4}
    \setcounter{section}{0}

    \section{Conclusion}

    In this dissertation, I have presented a series of computational methods to analyze spatial transcriptomics data. I began by developing Bento, a computational framework for subcellular analysis of spatial transcriptomics data. This work is one of the first to leverage the spatial resolution of imaging-based spatial transcriptomics data to study subcellular RNA localization. I demonstrated the utility of Bento by applying it to a variety of spatial transcriptomics datasets, including cardiomyocytes to study changes in RNA localization as scale. Unexpectedly, we found several genes mislocalized to the nucleus as a result of doxorubicin treatment, including RBM20 and CACNB2, suggesting that RNA localization is an underappreciated cell phenotype that has the potential to uncover functional biology. To lower the barrier to functional analysis of spatial transcriptomics datasets, I also created spotfish, a modular framework for decoding spatial imaging data. This framework is designed to be flexible, scalable, and interoperable with existing tools. I demonstrated the utility of spotfish by applying it to a 69-bit MERFISH dataset of U2-OS cells. The framework is aimed to be a community resource for building spatial transcriptomics pipelines, and I hope to eventually collaborate with the nf-core community to ensure that spotfish is well maintained and accessible to the community. 

    %%%%%%%%%%%%%%%%%%%%%%%%%%%%%%%%%%%%%%%%%%%%%%%%%%%%%%%%%%%%%%%%%%%%%%%%%%%%
    \section{Limitations and Future Directions}
    %%%%%%%%%%%%%%%%%%%%%%%%%%%%%%%%%%%%%%%%%%%%%%%%%%%%%%%%%%%%%%%%%%%%%%%%%%%%

    There are a great deal of challenges with the current generation of spatial transcriptomics data. During my graduate work, it was important to me that I focus on core problems that reveal fundamental biology, not a transient technical property of any one technology. For example, the most popular commercial spatial methods are slide-based capture assays paired with traditional sequencing for comprehensive transcriptome profiling. However, compared to imaging-based approaches which have single-molecule resolution, each spatial location on the assay captures several to tens of cells depending on the technology. This loss in fidelity has spawned an entire subfield of deconvolution methods, specifically to estimate properties of each spatial location such as the proportion of cell types, the expression of cells given predicted cell types, technical dropout of expression, etc. These techniques may be useful now, but are ultimately tied to a specific iteration of rapidly evolving spatial transcriptomic technologies. Instead, I chose to tackle problems initially hampered by the lack of tangible datasets; the recent availability of public datasets has indeed lowered the barrier to method development. The increasing throughput of new technologies such as Xenium from 10x Genomics and STOmics from BGI Genomics will only improve our ability to draw biological insights at the molecular resolution. Similarly, the imminent move towards multi-omics spatial imaging will enable us to capture more snapshots of the RNA life cycle than ever before.
    
    While the current functionality of Bento is limited to 2-dimensional spatial analysis, the obvious extension to 3 dimensions will enable subcellular analysis in biological systems more complex than monolayer cell cultures such as tissue slices and organoids. This will also open the door to exploring true physical molecular gradients in their natural 3 dimensions. While we showed its value to discover spatial subcellular domains, one can imagine gradient shifts between cells e.g. at the cell membrane, tight junctions, synapses etc. or even in the extracellular matrix characterizing cell signaling molecules. These are some of the functional biology questions waiting to be explored with spatial transcriptomics data. Bento is also capable of measuring morphological phenotypes; paired with the appropriate experimental design, spatial transcriptomics will be a powerful tool to interrogate the relationship between cell states, RNA localization, and cell morphology. These are especially relevant in developmental biology, neurological diseases, and cancer where changes to cell shapes and molecular condensates are frequently measured already. Algorithmic improvements will also require computational scalability, which I hope to address by interfacing with the global research community, including the Scverse Foundation\cite{virshupScverseProjectProvides2023} developers and the nf-core project members. Looking forward, this will manifest as integration with new open-source data standards, such as SpatialData and distributed computing with Dask. 
    

    %%%%%%%%%%%%%%%%%%%%%%%%%%%%%%%%%%%%%%%%%%%%%%%%%%%%%%%%%%%%%%%%%%%%%%%%%%%%
    \section{Closing thoughts}
    %%%%%%%%%%%%%%%%%%%%%%%%%%%%%%%%%%%%%%%%%%%%%%%%%%%%%%%%%%%%%%%%%%%%%%%%%%%%

    The field of spatial transcriptomics is still in its infancy, and there are many exciting opportunities for future computational work. I believe the most impactful innovations will come from other fields, such as computer vision and genomics. Deep learning has already made its mark in both fields to accomplish everything from self-driving cars to functional genomics with DNA large language models, with the potential to bridge the gap between imaging and sequencing. I am hopeful for the creativity in this field and am excited to see new applications beyond tissue atlases and drug screening platforms. I hope that my work will contribute to the growing body of open-source tools and resources for spatial transcriptomics, and that it will inspire others to do the same.

\end{dissertationepilogue}