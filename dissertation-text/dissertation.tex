\documentclass[11pt]{formatting-template}

% mathptmx is a Times Roman look-alike (don't use the times package)
% It isn't clear if Times is required. The OGS manual lists several
% "standard fonts" but never says they need to be used.
% \usepackage{mathptmx}

\usepackage[NoDate]{currvita}
\usepackage{array}
\usepackage{tabularx}
\usepackage{booktabs}
\usepackage{ragged2e}
\usepackage{microtype}
\usepackage[breaklinks=true,pdfborder={0 0 0}]{hyperref}
\usepackage{graphicx}
\AtBeginDocument{%
	\settowidth\cvlabelwidth{\cvlabelfont 0000--0000}
}

% OGS recommends increasing the margins slightly.
\increasemargins{.1in}

\usepackage[english]{babel}
\usepackage{blindtext}

\usepackage[super,sort&compress]{natbib}

\overfullrule5pt
%%%%%%%%%%%%%%%%%%%%%%%%%%%%%%%%%%%%%%%%%%%%%%%%%%%%%%%%%%%%%%%%%%%%%%%%%%%%%%%%
% Thesis Title & Author Information
\title{Decoding the genomic regulatory syntax driving notochord development}
\author{Michelle Franc Ragsac}
\degree{Bioinformatics \& Systems Biology}{Doctor of Philosophy}

% Thesis Committee Members
\chair{Professor Emma K. Farley}
\cochair{Professor Theresa Gaasterland} 
\committee{Professor Christopher Benner} % alphabetical order required
\committee{Professor Vineet Bafna}
\committee{Professor Xin Sun}
\degreeyear{2022}

%%%%%%%%%%%%%%%%%%%%%%%%%%%%%%%%%%%%%%%%%%%%%%%%%%%%%%%%%%%%%%%%%%%%%%%%%%%%%%%%
\begin{document}

% Begin with front matter and so forth
\frontmatter
\maketitle{}
\makecopyright{}
\makesignature{}

%%%%%%%%%%%%%%%%%%%%%%%%%%%%%%%%%%%%%%%%%%%%%%%%%%%%%%%%%%%%%%%%%%%%%%%%%%%%%%%%
% Dedication
%%%%%%%%%%%%%%%%%%%%%%%%%%%%%%%%%%%%%%%%%%%%%%%%%%%%%%%%%%%%%%%%%%%%%%%%%%%%%%%%
\begin{dedication}
	\setsinglespacing{}
	\parindent0pt\parskip\baselineskip{}
	\vskip0pt plus.25fil
	\begin{center}
		\textit{\textbf{Ang pamilya ay hindi isang mahalagang bagay lamang.\\ Ito ay ang lahat.}}

		$\cdots$

		This dissertation is dedicated to the family that fostered a warm,\\
		multi-generational home that valued love, curiosity, and education above all else.
		
		To my parents, \textit{Francisco Gregorio Ragsac} and \textit{Lea Reyes Ragsac},\\
		for paving the way for a better life in the United States.
		
		To my younger brother, \textit{Thomas Jonathan Ragsac},\\
		for spending time with me through playing video games because you had to. 

		To my aunt, \textit{Charleen Rodriguez Ragsac},\\
		for always keeping things loud and lively.
		
		Finally, to my grandparents, \textit{Hermogenes Riego de Dios Reyes} and \textit{Josefina Tardeo Reyes},\\
		for making sure I spent my days listening to classical music on the mandolin and eating comforting Filipino food during my youth.

		$\cdots$

		I would also like to dedicate this dissertation to \textit{Clarence Kuang-Le Mah},\\
		the nerdy best friend who got me interested in bioinformatics in the first place.\\
		I’d still walk from Camp Snoopy to the Village for you.
	\end{center}
\end{dedication}

%%%%%%%%%%%%%%%%%%%%%%%%%%%%%%%%%%%%%%%%%%%%%%%%%%%%%%%%%%%%%%%%%%%%%%%%%%%%%%%%
% Epigraph
%%%%%%%%%%%%%%%%%%%%%%%%%%%%%%%%%%%%%%%%%%%%%%%%%%%%%%%%%%%%%%%%%%%%%%%%%%%%%%%%
\begin{epigraph}
	\vskip0pt plus.5fil
	\vfil\vfil
	\setsinglespacing{}
	{ \flushright{
		It is not birth, marriage, or death, \\
		but gastrulation which is the most \\
		important time in your life.
		}
		\vskip\baselineskip{}
		\textit{Louis Wolpert}\par
	}
	\vfil\vfil\vfil
\end{epigraph}

%%%%%%%%%%%%%%%%%%%%%%%%%%%%%%%%%%%%%%%%%%%%%%%%%%%%%%%%%%%%%%%%%%%%%%%%%%%%%%%%
% Table of Contents, Figures, Tables, Etc.
%%%%%%%%%%%%%%%%%%%%%%%%%%%%%%%%%%%%%%%%%%%%%%%%%%%%%%%%%%%%%%%%%%%%%%%%%%%%%%%%
% Next comes the table of contents, list of figures, list of tables,
% etc. If you have code listings, you can use \listoflistings (or
% \lstlistoflistings) to have it be produced here as well. Same with
% \listofalgorithms.
\tableofcontents
\listoffigures
\listoftables

%%%%%%%%%%%%%%%%%%%%%%%%%%%%%%%%%%%%%%%%%%%%%%%%%%%%%%%%%%%%%%%%%%%%%%%%%%%%%%%%
% Acknowledgements
%%%%%%%%%%%%%%%%%%%%%%%%%%%%%%%%%%%%%%%%%%%%%%%%%%%%%%%%%%%%%%%%%%%%%%%%%%%%%%%%
\begin{acknowledgements}
	TBA
\end{acknowledgements}

%%%%%%%%%%%%%%%%%%%%%%%%%%%%%%%%%%%%%%%%%%%%%%%%%%%%%%%%%%%%%%%%%%%%%%%%%%%%%%%%
% Vita
%%%%%%%%%%%%%%%%%%%%%%%%%%%%%%%%%%%%%%%%%%%%%%%%%%%%%%%%%%%%%%%%%%%%%%%%%%%%%%%%
\begin{vita}
\noindent
\begin{cv}{}
\begin{cvlist}{}
	\item[2013--2017] B.S. in Bioengineering: Bioinformatics\\
		University of California, San Diego
	\item[2014] Undergraduate Research Technician, Albert La Spada Laboratory\\	
		University of California, San Diego
	\item[2014--2016] Undergraduate Research Assistant, Laboratory of Computational Genomics\\
		Scripps Institute of Oceanography 
	\item[2015--2016] Writing Studio Mentor, Sixth College Writing Studio\\
		University of California, San Diego
	\item[2016] FrontierLab Summer Research Intern, Laboratory of Protein Informatics\\
		Osaka University, Japan  
	\item[2016--2017] Undergraduate Research Assistant, Christopher Benner Laboratory\\
		University of California, San Diego
	\item[2017] Teaching Assistant\\Scripps Institute of Oceanography 
	\item[2017] Research Associate, Applications and Technology Services--Assays by Agena\\
		Agena Bioscience
	\item[2018] Research Associate, Research and Development--Molecular Tools\\
		Agena Bioscience 
	\item[2018--2022] Ph.D. in Bioinformatics \& Systems Biology,\\
		University of California, San Diego
	\item[2018] Teaching Assistant\\
		Scripps Institute of Oceanography
	\item[2020--2021] Teaching Assistant, School of Medicine\\
		University of California, San Diego 
	\item[2020--2021] Bootcamp Instructor, Bioinformatics \& Systems Biology Doctoral Program\\
		University of California, San Diego 
	\item[2020--2021] Finance Committee Chair, Graduate and Professional Student Association\\
		University of California, San Diego
	\item[2021--2022] Vice President of Financial Affairs, Graduate and Professional Student Association\\
		University of California, San Diego
\end{cvlist}
\end{cv}

%%%%% PUBLICATIONS %%%%%
% \publications{}
% \noindent``Distributions of Control Points in a System for Analysis of Stress
% Distribution'' IRE Transactions of the I.R.E\@. Professional Group on
% Automatic Control, vol. AC-7, pp 272--289, September 2005

%%%%% FIELDS OF STUDY %%%%% 
% This section was omitted within this thesis as it was optional.

\end{vita}

%%%%%%%%%%%%%%%%%%%%%%%%%%%%%%%%%%%%%%%%%%%%%%%%%%%%%%%%%%%%%%%%%%%%%%%%%%%%%%%%
% Dissertation Abstract 
%%%%%%%%%%%%%%%%%%%%%%%%%%%%%%%%%%%%%%%%%%%%%%%%%%%%%%%%%%%%%%%%%%%%%%%%%%%%%%%%
\begin{dissertationabstract}
	Embryonic development across all vertebrates begins upon the fertilization of an egg by a sperm cell to become a single-celled zygote. Embryogenesis continues with various stages of division to eventually make up an entire organism. The processes governing development are finely orchestrated and include many participants, such as genes involved in gene regulatory networks and non-coding regions of DNA, or enhancers, to regulate the expression of those genes. Defects or perturbations to this strictly regulated machinery can lead to various clinical conditions, such as congenital heart disease. Thus, deepening our understanding of embryogenesis may help us understand the mechanisms driving congenital abnormalities as well as the evolution of developmental pathways. One defining characteristic of all chordate embryos is the presence of a notochord during development. The notochord is a long, semi-rigid fibrous rod of mesodermal origin that provides structural support and serves as a signaling center to pattern the neighboring neural tube, paraxial mesoderm, and gut. A complete understanding of notochord structure and function during early and late life stages is thus essential to better understand congenital vertebral defects. For example, failure of vertebral notochord cells to transition to the nucleus pulposus, the cushioning between intervertebral discs of the spine, is associated with chordomas, slow-growing tumors formed from notochord cell remnants within the spine or the base of the skull. The ascidian \textit{Ciona intestinalis Type A} (\textit{Ciona}) is a marine organism that is evolutionarily similar to vertebrates. Through electroporation, \textit{Ciona} is readily amenable to high-throughput, high-resolution functional studies of cis-regulatory elements like enhancers in their native, whole-embryo context. To identify key notochord enhancers, I analyzed the importance of enhancer grammar–the transcription factor order, orientation, spacing, and binding affinity–in modulating notochord-specific expression. Next, I highlight the potential of single-cell RNA-sequencing to study the gene regulatory networks governing notogenesis and their relationship to congenital abnormalities. This body of work provides new insight into the regulatory processes governing notochord development, providing direction for future efforts to improve our understanding of notochord-based diseases across chordates. Finally, I highlight Open Educational Resources (OERs) I developed for Bioinformatics education, emphasizing accessibility and inclusion.
\end{dissertationabstract}

%%%%%%%%%%%%%%%%%%%%%%%%%%%%%%%%%%%%%%%%%%%%%%%%%%%%%%%%%%%%%%%%%%%%%%%%%%%%%%%%
% Main Text of the Dissertation
%%%%%%%%%%%%%%%%%%%%%%%%%%%%%%%%%%%%%%%%%%%%%%%%%%%%%%%%%%%%%%%%%%%%%%%%%%%%%%%%
\mainmatter{}

\begin{dissertationintroduction}
    As the defining structure of all chordates, the notochord plays a crucial role in signaling and coordinating development during embryogenesis. In most vertebrates, the notochord ossified into the vertebrae of the spine. However, the notochord persists throughout the life of some invertebrate chordates, such as amphioxus. This thesis dissertation focuses on understanding gene regulation in the notochord of the marine urochordate, \textit{Ciona intestinalis} (\textit{Ciona}), during embryonic development from the perspective of the genomic sequence and the perspective of active transcripts within this key structure.

    %%%%%%%%%%%%%%%%%%%%%%%%%%%%%%%%%%%%%%%%%%%%%%%%%%%%%%%%%%%%%%%%%%%%%%%%%%%%
    \section{Notochord development in \textit{Ciona intestinalis}}
    %%%%%%%%%%%%%%%%%%%%%%%%%%%%%%%%%%%%%%%%%%%%%%%%%%%%%%%%%%%%%%%%%%%%%%%%%%%%

    Chordates are animals belonging to the phylum Chordata, which includes vertebrates (subphylum Vertebrata), tunicates (subphylum Tunicata), and cephalochordates (subphylum Cephalochordata) \cite{holland2005}. The key defining characteristic of all chordates is the presence of a notochord during embryonic development \cite{stemple2004, holland2005, stemple2005, corallo2015, balmer2016, debree2018}. The notochord is a long, semi-rigid fibrous rod of mesodermal origin that provides structural support to the developing embryo along the anterior-posterior axis. The notochord also acts as a signaling center in the developing embryo, patterning structures such as the “neural tube” \cite{stemple2005, corallo2015, balmer2016}. A sheath of collagen proteins encases the notochord, allowing this flexible yet rigid structure to provide the basis for controlled mechanical support of Chordate organisms and protection for the neural tube \cite{corallo2015, stemple2004, stemple2005, balmer2016}.

    While some Chordates retain the notochord throughout life as their body’s primary axial support, in most vertebrates, the notochord becomes the nucleus pulposus of the intervertebral disc \cite{stemple2005, corallo2015, lawson2015, balmer2016}. Failure of vertebral notochord cells to transition to the nucleus pulposus is associated with chordomas. These slow-growing tumors form from notochord cell remnants within the spine or the base of the skull \cite{corallo2015, debree2018}. A complete understanding of notochord structure and function during early and late life stages is thus essential to better understand congenital neural tube and vertebral defects.

    As a close chordate relative to the vertebrates, the ascidian \textit{Ciona intestinalis Type A} or \textit{Ciona robusta} (\textit{Ciona}) stands as a longstanding model for studying organogenesis in a simple embryo \cite{dehal2002, delsuc2006, imai2006, satoh2014, satou2019, winkley2020}. For example, the \textit{Ciona} notochord consists of only 40 post-mitotic cells, and orthologs of many \textit{Ciona} notochord genes have known notochord expression in vertebrate embryos \cite{reeves2017, satoh2014, winkley2020}. Of the 40 notochord cells, 32 are grouped in the anterior of the body and compose the “primary” or “A-line” notochord. The remaining eight are located more posteriorly and form the “secondary” or “B-line” notochord \cite{satoh2014, winkley2020}. A-line and B-line refer to the conventional nomenclature denoting particular cell lineages in \textit{Ciona}. In the 4-cell \textit{Ciona} embryo, “A-lineage” and “B-lineage” cells are defined as the two cells on the vegetal side of the embryo, whereas the “a-lineage” and “b-lineage” cells are defined as the two cells on the animal side. The notochord thus forms from the vegetal A-line and B-line cells of the 4-cell \textit{Ciona} embryo \cite{satoh2014}.

    Within \textit{Ciona}, notochord precursor cells are defined as early as the eight-cell stage as the A4.1 and B4.1 blastomere pair in the developing anterior and posterior regions of the embryo, respectively \cite{corbo1997, satoh2014, yagi2004}. The A4.1 cells then divide to form the A5.1 and A5.2 blastomere pair at the onset of the 16-cell stage, which are precursors to the A-line notochord and the endoderm, nerve cord, trunk lateral cells, and muscle \cite{satoh2014}. On the other hand, the B4.1 cells divide to form the B5.1 and B5.2 blastomere pair and, through subsequent divisions from B5.1, divide into B6.1 and B6.2. Finally, the B6.1 blastomere descendant at the 32-cell stage will eventually develop into the B-line notochord and other mesenchymal and muscle cells \cite{corbo1997, satoh2014, yagi2004}. When gastrulation initiates at the 110-cell stage, the \textit{Ciona} embryo contains 16 primary and four secondary notochord precursor cells \cite{satoh2014}. Gastrulation is the stage at which the structure of the embryo changes from a single-layered blastula into a multiple-layered gastrula; thus, the notochord precursor cells coordinately invaginate as a monolayer over the primary gut, or archenteron \cite{rhee2005, winkley2020}. Following gastrulation is neurulation, the stage at which the embryonic neural plate develops and then forms the neural tube \cite{rhee2005, satoh2014}. At this stage, the notochord precursor cells in the \textit{Ciona} embryo divide for the last time to define the final set of notochord cells on the embryonic midline \cite{nakamura2012}. 

    %%%%%%%%%%%%%%%%%%%%%%%%%%%%%%%%%%%%%%%%%%%%%%%%%%%%%%%%%%%%%%%%%%%%%%%%%%%%
    \section{Elucidating the mechanisms regulating notogenesis}
    %%%%%%%%%%%%%%%%%%%%%%%%%%%%%%%%%%%%%%%%%%%%%%%%%%%%%%%%%%%%%%%%%%%%%%%%%%%%

    The massive developmental transitions during embryogenesis require accurate gene regulation to maintain and balance the differentiation process. One component of this machinery is the interactions between cis-acting DNA elements-such as promoters and enhancers-and regulatory transcription factors. With the first example discovered in the early 1980s, enhancers are short regions of DNA that contain transcription factor binding sites (TFBSs) which proteins proteins can bind to regulate gene transcription \cite{khoury1983, kvon2021, levine2010}. Additionally, enhancers are typically located distally from the gene promoter and are approximately 100 bp to 1,000 bp in length \cite{khoury1983, levine2010}. Interestingly, the presence of a collection of TFBSs alone is insufficient in encoding functional activity of a particular target gene. For example, only specific arrangements of binding sites can activate transcription. The overarching rules governing the functional arrangement of TFBSs within enhancers is termed "enhancer grammar." Enhancer grammar is the interplay between the syntax-the order, orientation, and spacing of TFBSs-and the binding affinity of TFBSs to confer expression of a given enhancer sequence \cite{arnone1997, jindal2021}. Despite the importance of enhancers and their known association with developmental defects and disease, we still do not entirely understand how an enhancer’s sequence encodes particular functions. In Chapter 1, I will discuss the investigation into a notochord enhancer governed by Zic, ETS, FoxA, and Brachyury (Bra) transcription factor binding sites \cite{farley2016, song2022}. Zic and ETS are co-expressed in the developing notochord of \textit{Ciona} and in other vertebrates and are important for notochord specification \cite{dykes2018,matsumoto2007a}. The preceding study which discovered a putative notochord grammar relying on Zic and ETS found an interplay between the syntax and affinity of the binding sites present, such that the organization could compensate for the affinity and vice versa \cite{farley2016}. In Chapter 1, I will discuss an enhancer screen in which I search for evidence of the Zic and ETS notochord enhancer grammar across the \textit{Ciona} genome and test for functionality in the \textit{Ciona} notochord through a pilot screen of 90 genomic elements at the embryonic tailbud stage. From this screen, we were able to identify nine notochord enhancers, finding that enhancer grammar is critical within one of these elements. We also identify that some enhancers contain TFBSs for Zic, ETS, FoxA, and Bra, and translate that this set of binding sites may be an important signature for Brachyury enhancers across Chordates \cite{song2022}. 
    
    Beyond the universal quality of containing transcription factor motifs, enhancer sequences can vary significantly in the location, length, and type of transcription factor binding sites present. Additionally, these changes can be even more dramatic as you compare across species \cite{villar2015, ward2012, wong2020}. However, studies have suggested that even with low sequence conservation, the function of specific enhancers may be conserved across species and that this function may be partly due to combinatorial action of conserved transcription factors \cite{claussnitzer2014, wong2020}. This may be because a single transcription factor across its homologs in multiple species may have similar binding properties and thus recognize identical DNA sequences \cite{peter2011, wong2020}. In Chapter 2, I will continue the discussion of the notochord enhancer grammar studied in Chapter 1 but in greater detail and at a larger scale across the \textit{Ciona} genome. Within this study, we develop improvements over our initial search of \textit{Ciona} genomic regions containing Zic and ETS, such as allowing for greater flexibility of the Zic binding site within a sequence window. We find 4,344 genomic regions that harbor at least one Zic binding site and two ETS binding sites and test these regions in a massively-parallel reporter assay. In Chapter 2, I describe our preliminary results which suggest that only 9\% of the tested elements are functional enhancers. Further study of this enhancer library will likely identify novel notochord enhancers and help us better understand how Zic and ETS encode notochord development through particular grammatical constraints. 
    
    Within Chapter 1 and Chapter 2, I conduct high-throughput screens of genomic elements within developing whole embryos to better understand how enhancers encode notochord-specific expression patterns. Nonetheless, understanding the underlying processes driving development also requires understanding how genes are expressed, primarily how these gene expression profiles differ across cells \cite{peter2011}. For instance, all cells in a developing embryo contain the same set of genes. However, different cells express different sets of these genes, leading to differences in expression and, thus, molecular function \cite{arnone1997, peter2011}. Technological advances have enabled the cataloging of global gene expression profiles of single cells using single-cell RNA-sequencing (scRNA-seq), allowing scientists to define the heterogeneity within cell populations during embryonic development \cite{klein2015a, macosko2015, olsen2018}. This new paradigm has allowed developmental biologists to identify precisely when and in which cell types genes controlling cell fate decisions are expressed \cite{klein2019}. Despite the availability of large cell-type atlases generated via scRNA-seq and other omics technologies, there is still much to be learned about gene regulatory networks. \textit{Ciona} is a particularly suitable model for understanding the transcriptional changes necessary for proper development due to its genomic and morphological simplicity and historical significance as a model organism for embryological studies. In Chapter 3, I discuss an initiative to develop a high-resolution, single-cell atlas of a gastrulating \textit{Ciona} embryo to understand notogenesis and the formation of other early structures.  
    
    %%%%%%%%%%%%%%%%%%%%%%%%%%%%%%%%%%%%%%%%%%%%%%%%%%%%%%%%%%%%%%%%%%%%%%%%%%%%
    \section{Training the next generation of bioinformaticians}
    %%%%%%%%%%%%%%%%%%%%%%%%%%%%%%%%%%%%%%%%%%%%%%%%%%%%%%%%%%%%%%%%%%%%%%%%%%%%

    In recent years, genomics technologies have become more high-throughput and affordable to all research groups, resulting in a boom in data available for all biomedical research areas. However, this also results in a backlog of data to analyze for those that conducted the experiments. Despite never receiving a formal education in computation, many researchers are then faced with the arduous task of learning how to run bioinformatics pipelines \cite{barone2017, stephens2015}. While computational courses have started being integrated into the standard curriculum for undergraduate Biology majors, there remains a need to support graduate students and other scientists that did not experience this shift in training for the field. In Chapter 4, I will discuss the pedagogical philosophy that drove the in-person, hybrid, and virtual Bioinformatics courses I taught at the University of California, San Diego. 

    %%%%%%%%%%%%%%%%%%%%%%%%%%%%%%%%%%%%%%%%%%%%%%%%%%%%%%%%%%%%%%%%%%%%%%%%%%%%
    \section{Conclusion}
    %%%%%%%%%%%%%%%%%%%%%%%%%%%%%%%%%%%%%%%%%%%%%%%%%%%%%%%%%%%%%%%%%%%%%%%%%%%%

    The massive developmental transitions during embryogenesis require accurate gene regulation to maintain and balance the differentiation process. In this dissertation, I present our approach to understanding regulation in the developing notochord by conducting high-throughput, whole embryo reporter screens to identify functional enhancers. I also present a novel, proof-of-concept package for performing flexible genomic searches of combinatorial arrangements of TFBSs. Additionally, I share our current understanding of \textit{Ciona} gastrulation and notogenesis from studying single-cell transcriptional expression profiles. Finally, I also discuss my contributions to Bioinformatics education.
    
\end{dissertationintroduction}

\chapter{Diverse logics encode notochord enhancers}
\label{chap:Diverse logics encode notochord enhancers}

The notochord is a key structure during chordate development. We have previously identified several enhancers regulated by Zic and ETS that encode notochord activity within the marine chordate \textit{\textit{Ciona} robusta} (\textit{Ciona}). To better understand the role of Zic and ETS within notochord enhancers, we tested 90 genomic elements containing Zic and ETS sites for expression in developing \textit{Ciona} embryos using a whole-embryo, massively parallel reporter assay. We discovered that 39/90 of the elements were active in developing embryos; however only 10\% (9/90) were active within the notochord, indicating that more than just Zic and ETS sites are required for notochord expression. Further analysis revealed notochord enhancers were regulated by three groups of factors: (1) Zic and ETS, (2) Zic, ETS and Brachyury (Bra), and (3) Zic, ETS, Bra and FoxA. One of these notochord enhancers, regulated by Zic and ETS, is located upstream of \textit{laminin alpha}, a gene critical for notochord development in both \textit{Ciona} and vertebrates. Reversing the ETS sites in this enhancer greatly diminishes expression, indicating that enhancer grammar is critical for enhancer activity. Strikingly, we find clusters of Zic and ETS binding sites within the introns of mouse and human \textit{laminin alpha-1} with conserved enhancer grammar. Our analysis also identified two notochord enhancers regulated by Zic, ETS, FoxA and Bra binding sites: the Bra Shadow (BraS) enhancer located in close proximity to the gene \textit{Bra}, and an enhancer located near the gene \textit{Lrig}. By creating a library of 45 million enhancer variants with the sequence, affinity and position of the Zic, ETS, FoxA and Bra sites fixed while all other nucleotides are randomized, we discover that these sites are necessary and sufficient for notochord expression. Zic, ETS, FoxA and Bra binding sites occur within the \textit{Ciona} Bra434 enhancer and vertebrate notochord Bra enhancers, suggesting a conserved regulatory logic. Collectively, this study deepens our understanding of how enhancers encode notochord expression, illustrates the importance of enhancer grammar, and hints at the conservation of enhancer logic and grammar across chordates. 

%%%%%%%%%%%%%%%%%%%%%%%%%%%%%%%%%%%%%%%%%%%%%%%%%%%%%%%%%%%%%%%%%%%%%%%%%%%%%%%%
\section{Introduction}
%%%%%%%%%%%%%%%%%%%%%%%%%%%%%%%%%%%%%%%%%%%%%%%%%%%%%%%%%%%%%%%%%%%%%%%%%%%%%%%%

Enhancers are genomic elements that act as switches to ensure the precise patterns of gene expression required for development \cite{levine2010}. Enhancers regulate the timing, locations and levels of expression by binding of transcription factors (TFs) to sequences within the enhancer known as transcription factor binding sites (TFBSs) \cite{heinz2010,liu2012a,small1992,spitz2012,swanson2010a}. This binding, along with protein-protein interactions, leads to recruitment of transcriptional machinery and activation of gene expression. While we understand that TFBSs regulate enhancers and mediate tissue-specific expression, we have limited understanding of how the sequence of an enhancer encodes a particular expression pattern and what combinations of binding sites within enhancers are able to mediate enhancer activity. Given that the majority of variants associated with disease and phenotypic diversity lie within enhancers \cite{maurano2012,tak2015a,visel2009}, it is critical that we understand how the underlying enhancer sequence encodes tissue-specific expression and what types of changes within an enhancer sequence can cause changes in expression, cellular identity and phenotypes. 

A set of grammatical rules that define how enhancer sequence encodes tissue-specific expression is an attractive idea first suggested almost 30 years ago \cite{arnone1997,barolo2016a,levo2014,thanos1995}. The hypothesis for grammatical rules is based on the fact that proteins and the enhancer DNA have physical properties. These physical constraints govern the interaction of proteins with DNA and could be read out within the DNA sequence at the level of TFBSs. Enhancer grammar is composed of constraints on the number, type, and affinity of TFBSs within an enhancer and the relative syntax of these sites (orders, orientations, and spacings) \cite{jindal2021}.

We previously identified grammatical rules governing notochord enhancers regulated by Zic and ETS TFBSs \cite{farley2016}. We found that there was an interplay between affinity and organization of TFBSs, such that organization could compensate for poor affinity and vice versa. Using these rules, we identified two novel notochord enhancers, Mnx and Bra Shadow (BraS). These enhancers use low-affinity ETS sites in combination with Zic sites to encode notochord expression \cite{farley2016}. Here, we focus on obtaining a deeper understanding of how enhancers regulated by Zic and ETS encode notochord expression. 

Zic and ETS are co-expressed in the developing notochord of the marine chordate \textit{Ciona} (Figure ~\ref{fig:1 zic ets expression}) and in vertebrates \cite{dykes2018,matsumoto2007a}. The notochord is a key feature of chordates and acts as a signaling center to pattern the neighboring neural tube, paraxial mesoderm, and gut \cite{herrmann1994,stemple2005}. Specification of the notochord by Brachyury (Bra), also known as T, is highly conserved across chordates \cite{chesley1935,chiba2009,wilkinson1990,yasuo1993}. Other conserved TFs important for activation of notochord gene expression include Zic, ETS, a TF downstream of FGF signaling, and FoxA \cite{ang1994,dal-pra2011,dykes2018,elms2004,imai2002,imai2002a,jose-edwards2015a,katikala2013,kumano2006,matsumoto2007a,miya2003,passamaneck2009a,schulte-merker1995,warr2008,weinstein1994,yagi2004,yasuo2007}.

Our study focuses on the marine chordate, \textit{\textit{Ciona} intestinalis type A},  also known as \textit{\textit{Ciona} robusta} (\textit{Ciona}), a member of the urochordates, the sister group to vertebrates \cite{delsuc2006}. Fertilized \textit{Ciona} eggs can be electroporated with many enhancers in a single experiment which allows for testing of many enhancers in whole, developing embryos \cite{davidson2006,farley2015a}. Furthermore, these embryos are transparent and have defined cell lineages, making it easy to image and determine the location of enhancer activity. These advantages, along with the fast development of \textit{Ciona} and the similarity of notochord development programs between \textit{Ciona} and vertebrates \cite{davidson2006,digregorio2020}, make it an ideal organism to study the rules governing notochord enhancers during development. 

Within the \textit{Ciona} genome, we found 1,092 elements containing one Zic site and at least two ETS sites within 30 bp upstream or downstream of the Zic site. We tested 90 of these for expression in developing \textit{Ciona} embryos. Only 10\% of these regions drive notochord expression. These notochord enhancers fall into three categories: enhancers containing Zic and ETS sites, ones with Zic, ETS and Bra sites, and ones with Zic, ETS, FoxA and Bra sites. Within enhancers containing Zic and ETS sites, the organization of sites is important for activity, indicating that grammatical constraints on Zic and ETS encode enhancer activity. We find that one of the Zic and ETS enhancers is near an important notochord gene, \textit{laminin alpha} \cite{veeman2008}. The orientation of binding sites within this \textit{laminin alpha} enhancer is critical for enhancer activity demonstrating the role of enhancer grammar.  We find similar clusters of Zic and ETS sites within the introns of \textit{laminin alpha-1} in both mouse and human. Strikingly, we find the same 12 bp spacing between the Zic and ETS conserved across all three species. Additionally, this study identifies two enhancers using a combination of Zic, ETS, FoxA, and Bra to encode notochord expression. One of these is the BraS enhancer. By creating a library of 45 million enhancer variants with the sequence, affinity and position of the Zic, ETS, FoxA and Bra sites fixed while all other nucleotides are randomized, we discover that these sites are necessary and sufficient for notochord expression. Other known Bra enhancers within \textit{Ciona} \cite{corbo1997} and vertebrates \cite{schifferl2021} also harbor this combination of TFs, suggesting that Zic, ETS, FoxA, and Bra is a common feature of Bra regulation in chordates. Collectively, our study finds that grammar is a key component of functional enhancers with signatures of this enhancer logic and grammar seen across chordates. 

\begin{figure}[h]
    \centering
    \includegraphics[scale=0.25]{2_figures-and-files/Fig1_ZicEts-Expression.png}
    \caption[Zic and ETS expression in the 110-cell stage embryo]{\textbf{Zic and ETS expression in the 110-cell stage embryo.} Co-expression of Zic and ETS is shown in purple and occurs in the notochord, a6.5 lineage, which gives rise to the anterior sensory vesicle and palps, and four mesenchyme cells shown in light purple. A schematic of the tailbud embryo shows the notochord and a6.5 cell types later in development. Dark coloring represents a6.5 and notochord lineages, and light coloring represents other tissues with expression of Zic and/or ETS.}
    \label{fig:1 zic ets expression}
\end{figure}

%%%%%%%%%%%%%%%%%%%%%%%%%%%%%%%%%%%%%%%%%%%%%%%%%%%%%%%%%%%%%%%%%%%%%%%%%%%%%%%%
\section{Results}
%%%%%%%%%%%%%%%%%%%%%%%%%%%%%%%%%%%%%%%%%%%%%%%%%%%%%%%%%%%%%%%%%%%%%%%%%%%%%%%%

\subsection{Searching for clusters of Zic and ETS sites within the \textit{Ciona} genome}

To better understand how Zic and ETS sites within enhancers encode notochord expression, we searched the \textit{Ciona} genome (KH2012) for clusters of Zic and ETS sites. To do this, we first identified Zic motifs in the genome. We defined Zic motifs using EMSA and enhancer mutagenesis data from previous studies (see methods for motifs) \cite{matsumoto2007a,takahashi1999,yagi2004}. Using the Zic site as an anchor, we searched the 30 bp upstream and downstream of  the Zic site for ETS sites, using the core motif \verb|GGAW| (\verb|GGAA| and \verb|GGAT|) to consider all ETS sites regardless of affinity \cite{lamber2008,wei2010}, as we have previously found that low-affinity ETS sites are required to encode notochord-specific expression \cite{farley2016}. This search identified 1,092 genomic regions approximately 68 bp in length. We define these regions as ZEE elements. 

\subsection{Testing ZEE genomic elements for enhancer activity in developing \textit{Ciona} embryos}

We selected 90 ZEE elements (Figure ~\ref{fig:supplement zee elements screened}1 and Table S1) and synthesized these upstream of a minimal promoter (bpFog) \cite{rothbacher2007,stolfi2015} and a transcribable barcode to conduct an enhancer screen (experiment outlined in Figure ~\ref{fig:2 enhancer screen schematic}A). Each enhancer was associated with, on average, six unique barcodes. Each different barcode is a distinct measurement of enhancer activity. We electroporated this library into fertilized \textit{Ciona} eggs. We collected embryos at the late gastrula stage (5.5 hours post fertilization, hpf) when notochord cells are developing \cite{jiang2007} and both Zic and ETS are expressed \cite{imai2004,winkley2021}. At this timepoint, we isolated mRNA and DNA. To determine that all the enhancer plasmids got into the embryos, we isolated the plasmids from the embryos and sequenced the DNA barcodes. We detected barcodes associated with all 90 ZEE elements from the isolated plasmids, indicating that all elements were tested for activity within the developing \textit{Ciona} embryos. 

We next wanted to see how many of the 90 ZEE elements act as enhancers to drive transcription. Active enhancers will transcribe the GFP and the barcode into mRNA. To find the functional enhancers, we isolated the mRNA barcodes from our electroporated embryos and sequenced them. We analyzed the sequencing data and measured the reads per million (RPM) for each barcode. To calculate an average RNA RPM for a given enhancer, we averaged the RPM for each RNA barcode associated with an enhancer. To normalize the enhancer activity to the differences in the amount of plasmid and therefore number of copies of the enhancer electroporated into embryos, we took the $log_2$ of the average enhancer RNA RPM divided by the DNA RPM for the same enhancer to create an enhancer activity score. Enhancer activity scores below zero are non-functional, while elements with scores above zero are considered functional enhancers. The highest activity score is around four. The experiment was repeated in biological triplicate and there was a high correlation between all three biological replicates (Figure ~\ref{fig:supplement notochord data qc}).

\begin{figure}[p]
    \centering
    \includegraphics[scale=.5]{2_figures-and-files/Fig2_Enhancer-Screen_Expression-Distribution.png}
    \caption[Screening Zic and ETS genomic elements in \textit{Ciona}]{\textbf{Screening Zic and ETS genomic elements in \textit{Ciona}.} \textbf{A.} Schematic of enhancer screen. 90 ZEE genomic regions, each associated with on average six unique barcodes were electroporated into fertilized \textit{Ciona} eggs. mRNA and plasmid DNA were extracted from 5.5 hpf embryos (tailbud embryo shown to highlight tissues with predicted expression). The mRNA and DNA barcodes were sequenced, and a normalized enhancer activity score was calculated for each enhancer by taking the $log_2$ of the mRNA activity for a given enhancer divided by the number of copies of the plasmid. \textbf{B.} Violin plot showing the distribution of enhancer activity. The Bra Shadow enhancer served as a positive control and is labeled. The red line indicates the cut-off for non-functional elements at zero. \textbf{C.} Same plot as (B), but with all 90 ZEE elements plotted as dots. Dots are colored by the results of an orthogonal screen, where we measured the GFP expression in at least 150 embryos to determine the location of expression (50 embryos per repeat). Enhancers driving notochord expression are shown in purple, enhancers with expression but no notochord expression are shown in orange. ZEE elements that do not drive expression are grey and untested enhancers are shown in white.}
    \label{fig:2 enhancer screen schematic}
\end{figure}

\subsection{Many genomic ZEE elements are not enhancers}

As an internal, positive control in our enhancer screen, we included the Bra Shadow (BraS) enhancer. This enhancer drives expression in the notochord and weak expression in the a6.5 lineage, both locations that express Zic and ETS \cite{farley2016}. The BraS enhancer activity score is 2.4 (Figure ~\ref{fig:2 enhancer screen schematic}B), indicating that our library screen is detecting functional enhancers. Thirty-nine of the ZEE elements act as enhancers in our screen, while fifty-one of the ZEE elements drove no expression. This suggests that genomic elements containing a single Zic site and at least two ETS sites are not sufficient to drive expression in the notochord. To further validate our sequencing data and to determine the tissue-specific location of the functional enhancers, we selected 20 non-functional elements and 24 functional enhancers from our screen to test by an orthogonal approach. Each of these ZEE elements were cloned upstream of a minimal bpFog promoter and GFP. We electroporated each enhancer into fertilized eggs and analyzed the GFP expression of these ZEE elements under the microscope at 8 hpf in at least 150 embryos across three biological replicates. Collectively, we analyzed expression of these elements in over 6,600 embryos with this orthogonal approach. 

All 20 ZEE elements defined as non-functional in our library drove no GFP expression, validating our enhancer activity score cut off that we defined for non-functional enhancers (Figure ~\ref{fig:2 enhancer screen schematic}C). In the 24 enhancers detected as functional within the enhancer screen, 92\% of these enhancers (22/24) showed GFP expression within the embryos when tested individually (Table S2). Nine ZEE elements drove expression in the notochord (Figure ~\ref{fig:supplement all notochord enhancers}3 and Table S3). Four of these enhancers are active almost exclusively in the notochord (ZEE10, 13, 20, 27). The remaining five are active in the notochord with additional expression in the endoderm and/or nerve cord (b6.5 lineage). Twelve of the ZEE enhancers drove varying levels of expression in the a6.5 lineage, which gives rise to the neural cell types called the anterior sensory vesicle and the palps, but only one drove expression exclusively in this cell type (ZEE22). Thirteen ZEE elements drove expression in one or more for the following cell types: the nerve cord (b6.5 lineage), mesenchyme, and endoderm. The expression patterns seen for these active enhancers are consistent with the expression patterns of Zic and ETS which are expressed in the muscle, endoderm, ectoderm, mesenchyme, notochord, a6.5 neural lineage and b6.5 neural cell types \cite{hudson2007,hudson2016,imai2006,picco2007,wagner2012} (Note, S1 discusses the expression patterns of the ZEE elements with notochord expression in more detail). The only cells to co-express both Zic and ETS are the notochord, a6.5, and a small number of mesenchyme cells (Figure ~\ref{fig:1 zic ets expression}). Therefore, enhancers under combinatorial control of Zic and ETS are likely to be active in the notochord and the a6.5 neural lineage \cite{ikeda2016,matsumoto2007a,wagner2012}. Collectively these results indicate that our enhancer screen accurately detects functional enhancers, and our tissue-specific analysis provides detailed expression patterns for these enhancers. 

\subsection{Elucidating the logic of the enhancers driving notochord expression}

Having seen that so few enhancers drive expression in the notochord, we were interested to better understand why these nine functional enhancers were active in the notochord. It is possible that they are functional due to the grammar of the Zic and ETS sites or because other TFBSs are required for notochord expression. To investigate these two hypotheses, we looked at the nine notochord enhancers in more detail. FoxA and Bra are two other TFs important for activation of notochord enhancers in chordates \cite{ang1994,casey1998,dal-pra2011,jose-edwards2015a,katikala2013,passamaneck2009a,wilkinson1990}. We therefore searched all 90 ZEE elements for FoxA and Bra sites. We used EMSA and crystal structure data to define \verb|TRTTTAY| as the FoxA motif \cite{katikala2013,li2017,passamaneck2009a} and \verb|TNNCAC| as the Bra motif \cite{casey1998,conlon2001,digregorio1999,dunn2009,muller1997}. 

\begin{figure}[h]
    \centering
    \includegraphics[scale=.55]{2_figures-and-files/Fig3_Notochord-Grammar-Groups.png}
    \caption[Combinations of transcription factors in ZEE enhancers that drive notochord expression]{\textbf{Combinations of transcription factors in ZEE enhancers that drive notochord expression.} Notochord-expressing ZEE elements were grouped by the combination of transcription factor binding sites present in each element. For each combination, an embryo schematic shows the overlapping region of expression for that given combination. Below the embryo schematic, the number of ZEE elements, the number of ZEE elements with notochord expression and schematics of the ZEE elements with notochord expression within each group. Zic (red), ETS (blue), FoxA (orange), and Bra (green) sites are annotated. Dark blue ETS sites have an affinity of greater than 0.5, light blue sites have an affinity of less than 0.5.}
    \label{fig:3 notochord enhancer groups}
\end{figure}

\subsection{The nine elements that drive notochord expression contain three different combinations of transcription factors}

Of the 90 genomic regions we tested, 42 had only Zic and ETS sites, 39 had Zic, ETS and Bra sites, 4 had Zic, ETS, FoxA, and Bra sites and 5 had Zic, ETS and FoxA sites. Ten percent of the enhancers containing only Zic and ETS sites drive notochord expression (4/42). Eight percent (3/39) of the enhancers containing Zic, ETS, and Bra drive notochord expression. None of the enhancers (0/5) containing Zic, ETS, and FoxA drive notochord expression, while fifty percent (2/4) of the enhancers containing Zic, ETS, FoxA and Bra are active in the notochord (Figure ~\ref{fig:3 notochord enhancer groups} and Figure ~\ref{fig:supplement annotated zee elements}). Thus, there are three groups of notochord enhancers that contain: (1) Zic and ETS sites alone, (2) Zic, ETS and Bra sites, or (3) Zic, ETS, FoxA, and Bra sites. Having found that only a few of the elements containing Zic and ETS sites alone were functional, we wanted to understand if the organization or grammar of sites within these enhancers was important.

\begin{figure}[p]
    \centering
    \includegraphics[scale=.52]{2_figures-and-files/Fig4_Laminin-alpha-Enhancer.png}
    \caption[Zic and ETS grammar encodes a notochord \textit{laminin alpha} enhancer]{\textbf{Zic and ETS grammar encodes a notochord \textit{laminin alpha} enhancer.} \textbf{A.} Embryo electroporated with the Lama enhancer (ZEE13); GFP expression can be seen in the notochord. \textbf{B.} Embryo electroporated with Lama -E3, where ETS3 was mutated to be non-functional; no GFP expression detected. \textbf{C.} Embryo electroporated with Lama -Z, where the Zic was mutated to be non-functional; no GFP expression detected. \textbf{D.} Embryo electroporated with Lama RE3, where the sequence of ETS3 was reversed; no GFP expression detected. Comparable results were seen when ETS1 was reversed. \textbf{E.} Schematics of Zic and ETS clusters near \textit{laminin alpha}  in the genome of \textit{Ciona}, mouse, and human. All three \textit{laminin alpha}  clusters have a spacing of 12 bp between an ETS and Zic site and all contain non-consensus ETS sites. ETS site affinity scores are noted above each site. Dark blue ETS sites have an affinity of greater than 0.5, light blue sites have an affinity of less than 0.5.}
    \label{fig:4 laminin alpha enhancer}
\end{figure}

\subsection{Zic and ETS enhancer grammar encodes notochord \textit{laminin alpha}  expression}

Four enhancers containing Zic and ETS sites only (ZEE13, ZEE20, ZEE27 and ZEE85)  drive notochord expression. ZEE13, ZEE20 and ZEE27 drive expression only in the notochord and have similar levels of expression. ZEE85 drives expression predominantly in the nerve cord (b6.5 lineage) with weak notochord expression. ZEE20, ZEE27, and ZEE85 are not in close proximity to known notochord genes, though it is possible that these elements regulate notochord genes further away. The ZEE13 enhancer is located close to \textit{laminin alpha} , which is critical for notochord development \cite{veeman2008} (Figure ~\ref{fig:4 laminin alpha enhancer}A). Given the proximity of this notochord-specific enhancer to \textit{laminin alpha} , we decided to focus further analysis on this enhancer, which we renamed the Lama enhancer. Notably, this enhancer contains three ETS sites. To determine the affinity of these sites, we used Protein Binding Microarray data (PBM) for mouse ETS-1 \cite{wei2010}, as the binding specificity of ETS is highly conserved across bilaterians \cite{nitta2015,wei2010}. The consensus highest-affinity site has a score of 1.0, and all other 8-mer sequences have a score relative to the consensus. The Lama enhancer contains two ETS sites with exceptionally low affinities of 0.10, or 10\% of the maximal binding affinity, while the most distal ETS site is a high-affinity site (0.73). 

To determine if the Zic site and ETS sites are important for enhancer activity, we made a point mutation to ablate the ETS3 site, which we chose because it has the highest affinity (Figure ~\ref{fig:4 laminin alpha enhancer}B, Figure ~\ref{fig:supplement manipulated enhancers}A, and Table S4). This led to a complete loss of notochord activity indicating that this ETS site contributes to enhancer activity. Similarly, ablation of the Zic site results in complete loss of enhancer activity, indicating that both Zic and ETS sites are necessary for activity of this Lama enhancer (Figure ~\ref{fig:4 laminin alpha enhancer}C, Figure ~\ref{fig:supplement manipulated enhancers}A, and Table S4). We did not ablate the low affinity ETS sites of the Lama enhancer. Previously, we saw that the organization of sites within enhancers, a component of enhancer grammar, is critical for enhancer activity in both the Mnx and Bra enhancers. To see if enhancer grammar is important for activity within the Lama enhancer, we altered the orientation of sites within this enhancer and measured the impact on enhancer activity. Reversing the orientation of the first ETS site, which has an affinity of 0.10, led to a dramatic reduction in notochord expression, suggesting the orientation of this ETS site is important for enhancer activity. Similarly, reversing the orientation of the third ETS site (Lama RE3), which has an affinity of 0.73, also causes a loss of notochord expression (Figure ~\ref{fig:4 laminin alpha enhancer}D, Figure ~\ref{fig:supplement manipulated enhancers}A, and Table S4). These two manipulations demonstrate that the orientation of these ETS sites within this enhancer is important for activity, and thus, that there are some grammatical constraints on the \textit{Ciona} Lama enhancer. It is likely that grammar is an important feature of enhancers regulated by Zic and ETS, as we have previously seen similar grammatical constraints on the orientation and spacing of binding sites within the Mnx and BraS enhancer, and because so few genomic elements containing these sites are functional \cite{farley2016}. 

\subsection{Vertebrate \textit{laminin alpha-1} introns contain clusters of Zic and ETS with conserved spacing.}

The expression of laminin in the notochord is highly conserved between urochordates and vertebrates \cite{reeves2017,scott2004,veeman2008}. Indeed, laminins play a vital role in both urochordate and vertebrate notochord development, with mutations in laminins or components that interact with laminins causing notochord defects \cite{machingo2006,parsons2002,pollard2006}. The \textit{Ciona} \textit{laminin alpha} is the ortholog of the vertebrate \textit{laminin alpha 1/3/5} family. We therefore sought to determine if we could find a similar combination of Zic and ETS sites in proximity to vertebrate laminin genes, as both Zic \cite{dykes2018,warr2008} and ETS \cite{barnett1998,olivera-martinez2012} are important in vertebrate notochord development. Strikingly, we find a cluster of Zic and ETS sites within the intron of both the mouse and human \textit{laminin alpha-1} genes. The affinity of the ETS sites in all three species is also far from the consensus: the human cluster contains three ETS sites of 0.12, 0.17 and 0.25 affinity, while the putative mouse enhancer contains fewer, but higher-affinity, ETS sites (Figure ~\ref{fig:4 laminin alpha enhancer}E). We have previously seen that the spacing between Zic and adjacent ETS sites affects levels of expression, with spacings of 11 and 13 bp seen between ETS and Zic sites in the BraS enhancer and Mnx enhancer, respectively \cite{farley2016}. In line with this observation, the \textit{laminin alpha-1} clusters in mouse and human and the \textit{Ciona} Lama enhancer have a 12 bp spacing between the ETS and adjacent Zic site in all three species, suggesting that such spacings (11 to 13 bp) are a feature of some notochord enhancers regulated by Zic and ETS. The conservation of this combination of sites, the low-affinity ETS sites, and the conserved spacing hints at the conservation of enhancer grammar across chordates.

\subsection{The Zic, ETS, FoxA and Bra regulatory logic encodes notochord enhancer activity}

The group of genomic elements most enriched in notochord expression was the group containing Zic, ETS, FoxA and Bra binding sites, with two of the four driving notochord expression. Both of these enhancers are located near genes expressed in the notochord \cite{reeves2017}. The first was our positive control BraS, while the second enhancer is in proximity of the \textit{Lrig} gene. Both of these enhancers drive strong notochord expression along with some neural a6.5 expression. 

We previously identified the BraS enhancer through a search for rules governing Zic and ETS grammar that included number and type of TFBSs, along with the affinity, spacing, and orientation of TFBSs \cite{farley2016}. The BraS enhancer contains a Zic and two low-affinity ETS sites (0.14 and 0.25). We previously saw that changing the orientation of the lowest affinity ETS site, located 11 bp from the Zic site, leads to loss of expression, indicating that there are grammatical constraints on this enhancer and that the 0.14 affinity ETS site is important for expression \cite{farley2016}. To further confirm the role of the Zic and two ETS sites within BraS, we ablated these three sites (Zic and both ETS sites) with point mutations; this leads to complete loss of expression, demonstrating that these sites are necessary for notochord expression (Figure ~\ref{fig:5 brachyury shadow dissection}B, Figure ~\ref{fig:supplement manipulated enhancers}B, and Table S4). To test if these sites are sufficient for notochord expression, we created a library of 24.5 million variants in which  the Zic and two ETS sites were kept constant in sequence, affinity, and position while all other nucleotides were randomized. We electroporated this library into embryos and counted GFP expression in 8hpf embryos. BraS has notochord expression in 73\% of embryos, while the ZEE-randomized BraS enhancer (BraS rZE) has notochord expression in only 28\% of embryos. Thus, BraS rZE drives expression within the notochord in significantly fewer embryos than BraS, indicating that there are other sites within the enhancer that are also important for tissue-specific expression (Figure ~\ref{fig:5 brachyury shadow dissection}C, Figure ~\ref{fig:supplement manipulated enhancers}B, and Table S4). This experiment highlights the importance of understanding sufficiency in addition to necessity of sites.

Two obvious candidates for additional functional sites within BraS are the FoxA and Bra sites, which we detected in this enhancer. Both FoxA and Bra are TFs known to regulate notochord enhancers in urochordates and vertebrates \cite{ikeda2016,jose-edwards2015a,kumano2006,lolas2014,passamaneck2009a,reeves2021}. To test if the Bra and FoxA sites contribute to expression we ablated these sites. Ablating the Bra site within BraS leads to a significant reduction in expression, as does ablating the FoxA site (Figure ~\ref{fig:5 brachyury shadow dissection}D and E, Figure ~\ref{fig:supplement annotated zee elements}B, and Table S4). These manipulations suggest that all five sites (Zic, FoxA, Bra, and two ETS sites) are necessary for enhancer activity, and that all four TFs contribute to the activity of BraS.  

To test if the Zic, two ETS, FoxA and Bra sites are sufficient for notochord expression, we created another BraS randomization library with 45 million variants in which the Zic, ETS, FoxA, and Bra (ZEFB) sites were fixed in sequence, position and affinity and all other nucleotides within the enhancer were randomized. When we electroporated this library into \textit{Ciona}, the number of embryos showing notochord expression between the BraS ZEFB-randomized library (BraS rZEFB) and BraS WT was not significantly different (73\% BraS vs 62\% BraS rZEFB) (Figure ~\ref{fig:5 brachyury shadow dissection}F, Figure ~\ref{fig:supplement manipulated enhancers}B, and Table S4), suggesting that these five sites together are sufficient to drive notochord expression in the BraS enhancer. While there is no significant difference in the number of embryos with notochord expression between the BraS rZEFB and BraS enhancers, we noticed that expression in the notochord was slightly weaker for BraS rZEFB (p=0.03) (Figure ~\ref{fig:supplement annotated zee elements}C), suggesting that other elements within the randomized region may further augment the levels of notochord expression. We also noted that significantly fewer embryos drive expression in the a6.5 lineage in the BraS rZEFB relative to the BraS enhancer (14\% vs 32\% of embryos respectively, p<0.01) (Figure ~\ref{fig:supplement annotated zee elements}D) suggesting that sequences within the randomized region are important for the neural a6.5 expression. Studies of enhancers often stop when mutation experiments demonstrate a TF is necessary for enhancer activity. However, this falls short of a full understanding of enhancers. Our results highlight that finding necessary sites is not enough to identify the regulatory logic of an enhancer. These necessity and sufficiency experiments have uncovered a deeper understanding of the BraS enhancer, namely that it is regulated by Zic, ETS, FoxA, and Bra.

\begin{figure}[p]
    \centering
    \includegraphics[scale=.74]{2_figures-and-files/Fig5_BraS-Dissection.png}
    \caption[Zic, ETS, FoxA, and Bra may be a common regulatory logic for \textit{Brachyury} enhancers]{\textbf{Zic, ETS, FoxA, and Bra may be a common regulatory logic for \textit{Brachyury} enhancers.} \textbf{A.} Embryo electroporated with the Bra Shadow (BraS) enhancer; GFP expression can be seen in the notochord. \textbf{B.} Embryo electroporated with BraS -ZEE, where the Zic and two ETS sites were mutated to be non-functional; no GFP expression was detected. \textbf{C.} Embryo electroporated with BraS rZE, where the Zic and two ETS sites were fixed, and all other nucleotides were randomized; GFP expression was greatly diminished. \textbf{D.} Embryo electroporated with BraS -Bra, where the sequence of Bra was mutated to be non-functional; GFP expression was greatly diminished. \textbf{E.} Embryo electroporated with BraS -FoxA, where the sequence of FoxA was mutated to be non-functional; GFP expression was greatly diminished. \textbf{F.} Embryo electroporated with BraS rZEFB, where the Zic, two ETS, FoxA, and Bra sites were fixed, and all other nucleotides were randomized; GFP expression can be seen in the notochord \textbf{G-I.} Schematics of Zic (red), ETS (blue), FoxA (orange), and Bra (green) clusters near Bra in the genomes of \textit{Ciona} and mouse. }
    \label{fig:5 brachyury shadow dissection}
\end{figure}

\subsection{Zic, ETS, Bra and FoxA may be a common regulatory logic for \textit{Ciona} \textit{Brachyury} enhancers}

The first and most well-studied Bra enhancer is the Bra434 enhancer \cite{corbo1997,fujiwara1998}, which drives strong expression in the notochord (Figure ~\ref{fig:supplement bra434 annotated}A).  Bra434 enhancer contains Zic, ETS, FoxA, and Bra sites; ablating these sites within this enhancer lead to reduced expression, suggesting that these sites contribute to enhancer activity \cite{reeves2021,shimai2022}. There are different reports regarding the number and location of Zic, ETS, FoxA, and Bra sites within the Bra434 enhancer depending on the method used to define sites \cite{corbo1997,shimai2022}. Here we annotate the Bra434 enhancer using crystal structure data, enhancer mutagenesis data, EMSA and PBM data \cite{casey1998,conlon2001,digregorio1999,dunn2009,katikala2013,lamber2008,li2017,matsumoto2007a,muller1997,passamaneck2009a,takahashi1999,wei2010,yagi2004}. 

Our approach identifies two Zic sites, six low-affinity ETS sites, three FoxA sites, and eight Bra sites (Figure ~\ref{fig:5 brachyury shadow dissection}G and Figure ~\ref{fig:supplement bra434 annotated}B). Of these TFs, the least information is available regarding Zic; thus, it is possible that there are other more degenerate Zic sites that may be identified in future studies \cite{corbo1997,fujiwara1998,reeves2021,shimai2022}. Bra434 has stronger expression in the notochord than BraS and this may be due to the longer length of the Bra434 enhancer and the presence of more Zic, ETS, FoxA and Bra sites within Bra434 relative to BraS enhancer. Having seen that clusters of Zic, ETS, FoxA, and Bra are important in the BraS and Bra434 enhancers, we next wanted to see if this logic is found in Bra enhancers in vertebrates.

\subsection{Vertebrate notochord enhancers contain clusters of Zic, ETS, Fox and Bra, suggesting this is a common logic for regulation of \textit{Brachyury} expression in the notochord}

In mouse, the most well-defined notochord enhancer to date is within an intron of T2, 38kb upstream of T, which is the mouse ortholog of Bra (Figure ~\ref{fig:5 brachyury shadow dissection}H) \cite{schifferl2021}. This mouse T enhancer is required for \textit{Bra/T} expression, notochord cell specification and differentiation \cite{schifferl2021}. Homozygous deletion of this \textit{Bra/T} enhancer in mouse leads to reduction of \textit{Bra/T} expression, a reduction in the number of notochord cells, and halving of tail length. Bra/T and FoxA binding sites have previously been identified within this enhancer \cite{schifferl2021}. We find that this mouse \textit{Bra/T} enhancer also contains Zic and ETS binding sites. Within this enhancer there are 12 ETS sites; 11 of these have affinities ranging from 0.09-0.14, while one site has an affinity of 0.65, indicating that this enhancer contains low-affinity ETS sites. 

As we saw with the \textit{Ciona} BraS and Bra434 enhancer, typically there are multiple enhancers that all regulate the same or similar patterns of expression \cite{frankel2010,hong2008,perry2010}. This is thought to confer the transcriptional robustness required for successful development \cite{antosova2016,frankel2010,osterwalder2018,perry2010}. Following this logic, we continued to search the mouse \textit{Bra/T} region to see if we could find other putative notochord enhancers that may regulate \textit{Bra/T}. We identified a region located 2kb downstream of T that contains a cluster of Zic, low-affinity ETS (0.11-0.12), FoxA and Bra sites (Figure ~\ref{fig:5 brachyury shadow dissection}I). This putative enhancer occurs within an open chromatin region in mouse E8.25 notochord cells \cite{pijuan-sala2020}, suggesting this may be another mouse T enhancer. Similarly in zebrafish, a notochord enhancer located 2.1kb upstream of the Bra ortholog \textit{ntl} \cite{harvey2010} also contains a cluster of Zic, ETS, FoxA, and Bra sites (Table S6). The presence of these four TFs in Ciona, zebrafish, and mouse Bra enhancers suggests that the use of Zic, ETS, FoxA and Bra could be a common enhancer logic regulating expression of the key notochord-specification gene Bra in chordates.

%%%%%%%%%%%%%%%%%%%%%%%%%%%%%%%%%%%%%%%%%%%%%%%%%%%%%%%%%%%%%%%%%%%%%%%%%%%%%%%%
\section{Discussion}
%%%%%%%%%%%%%%%%%%%%%%%%%%%%%%%%%%%%%%%%%%%%%%%%%%%%%%%%%%%%%%%%%%%%%%%%%%%%%%%%

In this study we sought to understand the regulatory logic of notochord enhancers by taking advantage of high-throughput studies within the marine chordate \textit{Ciona}. Within the \textit{Ciona} genome, there are 1,092 genomic regions containing a Zic site within 30 bp of two ETS sites. We tested 90 of these ZEE genomic regions for expression in developing \textit{Ciona} embryos. Surprisingly, only nine of the regions drove notochord expression. Among these nine, we identified a \textit{laminin alpha} enhancer that was highly dependent on grammatical constraints for proper expression. We found a similar cluster of Zic and ETS sites within the intron of the mouse and human \textit{laminin alpha-1} gene; strikingly, these clusters and the \textit{Ciona} laminin enhancer have the same spacing between the Zic and ETS sites. Within the BraS enhancer, although Zic and ETS are necessary for enhancer activity, randomization of the BraS enhancer keeping only the Zic and ETS sites constant in a sea of 24.5 million variants reveals that these sites are not sufficient for notochord activity. FoxA and Bra sites are also necessary for notochord expression. Indeed, creating a library of 45 million BraS variants in which all five TFBSs are kept constant in position, and affinity while all other nucleotides are randomized leads to notochord expression in a similar proportion of embryos as the WT BraS, which indicates these sites are sufficient for notochord expression . We find that the combination of Zic, ETS, FoxA, Bra occurs within other Bra enhancers in \textit{Ciona} and vertebrates suggesting this combination of TFs may be a common logic regulating Bra expression. Our study identifies new developmental enhancers, demonstrates the importance of enhancer grammar within developmental enhancers and provides a deeper understanding of the regulatory logic governing Bra. Our findings of the same clusters of sites within vertebrates hint at the conserved role of grammar and logic across chordates. 

\subsection{Very few genomic regions containing Zic and two ETS sites are functional enhancers}

Our analysis of 90 genomic elements all containing at least one Zic site in combination with two ETS sites strikingly demonstrated that clusters of sites are not sufficient to drive expression. Only 39 of the 90 (43\%) elements tested drove any expression, and even more surprisingly, only 15 of these drove expression in lineages that co-express Zic and ETS, namely the a6.5 (anterior sensory vesicle and palps) and/or notochord. These findings indicate that searching for clusters of TFs is only minimally effective in identification of enhancers and suggests that the organization of sites is also important for rendering a cluster of binding sites a functional enhancer. Our findings are in agreement with the work from King et al., that found only 28\% of the genomic elements they tested for enhancer function in ES cells drove enhancer activity, despite the fact that these genomic elements contain TF motifs and bound these TFs in ChIP-seq assays \cite{king2020a}. Our study and King et al. suggest that having motifs, or even TF binding is not sufficient to drive expression and suggests that the grammar of these sites is critical for rendering a cluster of TFBSs a functional enhancer \cite{king2020a}. 

\subsection{Grammar is a key constraint of the Lama and BraS enhancers} 

Zic and ETS are necessary for activity of the Lama enhancer. Within the Lama enhancer, the orientation of binding sites relative to each other was critical for expression, providing evidence that enhancer grammar is a critical feature of functional enhancers regulated by Zic and ETS. Flipping the orientation of either the first or last ETS sites relative to the Zic site led to loss of enhancer activity in the \textit{Ciona} Lama enhancer. This mirrors the results of flipping the orientation of the ETS sites within the BraS enhancer \cite{farley2016}. \textit{Laminin alpha} is a key gene involved in notochord development in both \textit{Ciona} and vertebrates \cite{pollard2006,veeman2008}. Intriguingly, we find that both the human and mouse \textit{laminin alpha-1} have introns that harbor a similar cluster of Zic and ETS sites to those seen within \textit{Ciona}. There is a conservation of 12 bp spacing between the Zic and ETS site across all three chordate enhancers, similar to the spacing we have observed between Zic and ETS sites within the notochord enhancers Mnx and BraS \cite{farley2016}. We note that the vertebrate regions do not drive notochord expression in \textit{Ciona}. It possible that grammar is subtly tweaked between different species. Alternatively, the lack of activity could be due to promoter incompatibility across species, as in our assay we tested the mouse and human Lama enhancers with a \textit{Ciona} promoter. Reporter assays within mouse embryos could further investigate the functionality of the mouse and human Lama putative enhancers and the role of the 12 bp spacing within these elements.   

\subsection{Necessity of sites does not mean sufficiency--a deeper understanding of the BraS enhancer}

Our study of the BraS enhancer highlights the importance of testing sufficiency of sites to investigate if we fully understand the regulatory logic of an enhancer. We previously demonstrated that reversing the orientation of an ETS site led to loss of notochord expression in the BraS enhancer. Here, in this study, we show via point mutations that both Zic and ETS sites are required for enhancer activity. However, randomization of the BraS enhancer to create 24.5 million variants in which only the Zic and ETS sites are constant demonstrates that these sites are not sufficient for enhancer activity, as the randomized BraS enhancer (BraS rZE) only drives notochord expression in less than half the number of embryos as the BraS enhancer. Having discovered that Zic and ETS alone were not sufficient, we find that both FoxA and Bra sites also contribute to the enhancer activity. In a library of 45 million variants in which the Zic, ETS, Bra and FoxA sites are kept constant in sequence, affinity and position within a randomized backbone (BraS rZEFB), we see no significant difference in the number of embryos with notochord expression. This indicates that these five sites are necessary and sufficient for enhancer activity. However, the neural expression seen with the BraS enhancer appears to depend on some features within the randomized backbone, as the ZEFB library drives significantly less neural expression. We also note that the BraS rZEFB drives slightly weaker levels of notochord expression. These findings illustrate that enhancers are densely encoded with many features which contribute to expression. This is in line with recent work suggesting that enhancers contain far more regulatory information that previously appreciated \cite{fuqua2020}. It is possible that degenerate Zic, ETS, FoxA, or Bra sites could be present or novel TFBS are also contributing to this logic. Further analysis conducting MPRAs with these two libraries (BraS rZE and BraS rZEFB) will determine what other features are contributing to  notochord and neural expression. Sufficiency experiments are rarely done, and we are unaware of another study that has tested sufficiency across the entirety of an enhancer in developing embryos. However, our experiments demonstrate the importance of testing sufficiency to determine all the features contributing to enhancer function and illustrate the dense encoding of regulatory information within enhancers. 

\subsection{Partial grammatical rules can provide signatures that identify enhancers, but improved understanding could lead to more accurate predictions}

We were able to find the BraS enhancer using grammatical constraints on organization and spacing between Zic and ETS site and affinity of ETS sites \cite{farley2016}. Interestingly, we did not have all the features required for enhancer activity. As such, this suggests that partial knowledge of grammatical constraints, or partial signatures of grammar could be used to identify functional enhancers. Our previous strategy searched for these grammatical constraints in proximity of known notochord genes, which may be why we were successful in identification of the Mnx and BraS enhancer with only partial grammar rules. Understanding the dependency between all features within an enhancer will likely enable greater success in identification of functional regulatory elements, as current genomic screens have shown limited success of identifying functional enhancers through epigenetic markers and transcription factor binding sites alone \cite{king2020a}. Until then, our current knowledge of grammatical constraints may still be useful for pointing us towards putative enhancers. 

\subsection{Zic, ETS, FoxA, and Bra may be a common logic upstream of \textit{Brachyury} in chordates}

The Bra434 enhancer also contains the same combination of sites as the BraS enhancer; therefore, it is possible that this is a common logic for regulating Bra. Interestingly, we find these sites within mouse and zebrafish Bra enhancers \cite{harvey2010,schifferl2021}. While there are differences in expression dynamics of these factors in vertebrates and ascidians, it is striking to see this combination of sites in validated notochord enhancers across these species. Indeed, our study in both the laminin enhancers and Bra enhancers provides hints of a conserved regulatory logic across chordates, although future tests of these putative enhancers within mouse are required to see if these are truly conserved enhancers with similar grammar signatures. Our study focuses on conservation of grammatical signatures rather than sequence conservation. A recent study searching for conserved enhancers in syntenic regions suggests that there may be much more conservation of enhancer function than expected based on sequence conservation \cite{wong2020}. Our approach searching for grammatical signatures rather than sequence conservation may allow for identification of such functionally conserved enhancers.

\subsection{Approaches to understanding dependency grammar of notochord expression}

Searching for grammatical rules governing enhancers requires comparison of functional enhancers with the same features. Although we thought we had the same features in all 90 regions, we actually had at least three distinct types of enhancers within our screen. This illustrates a common problem in mining genomic data for patterns, as the assumption that we are comparing like with like is often an incorrect one. Other screens mining genomic elements have hit similar roadblocks, with only a few functional genomic examples being uncovered and thus limiting the ability to find grammatical rules \cite{king2020a}.  To uncover the grammatical constraints on enhancers, we need to not only understand the number and types of sites within an enhancer, but also the dependency between these sites, such as affinity, spacing, and orientation \cite{jindal2021}. 

Massively or gigantic parallel reporter assays with increased size and complexity and that combine both synthetic enhancers and genomic elements will likely be required to pinpoint the rules governing enhancer activity within genomes. However, integrating synthetic screens with genomic screens is a major challenge as synthetic screens often have limited application within the context of the genome \cite{king2020a}. Another approach is to study entirely random sequences for enhancer activity, which has been done in the context of promoters in bacteria and yeast \cite{yona2018,deboer2020a}. Indeed, the conclusions of these studies mirror our own findings that grammar and low-affinity sites are critical components of functional regulatory elements. However, as 83\% of the random sequences within yeast drove expression, it is unclear how well random sequences mirror the regulatory landscape within the genome that has been shaped by evolutionary constraints over millions of years. Nonetheless, testing random sequences within the context of developing embryos could provide another source of data to understand how enhancers encode tissue-specific expression \cite{galupa2022}. In the future, integration of genomic regions, synthetic designed, and random sequences will contribute to our understanding of enhancer grammar. Despite the complexity of studying enhancers in developing embryos, our study demonstrates that enhancer grammar is critical for encoding notochord activity and our observation of the same logics and grammar signatures in both \textit{Ciona} and vertebrates hints at conservation of these grammatical constraints across chordates. 

\subsection{Limitations of the study}

In this study, we screened 90 ZEE elements for functionality; however, only 10\% were active in the notochord. We anticipate that discovering more notochord enhancers regulated by Zic, ETS, or regulated by Zic, ETS, FoxA, and Bra could better inform our understanding of notochord grammar. Towards this end, testing all 1,092 ZEE elements we identified within the \textit{Ciona} genome could strengthen this study. However, this would likely only yield 100 notochord enhancers, which would still not be enough to define grammatical rules. As discussed above, combining assays of genomic regions with synthetic and random enhancer screens could help gain enough data to determine the grammar of notochord enhancers. 

Another limitation relates to our identification of conserved enhancer logic and grammar across chordates. While we identified similar signatures with the Lama enhancers in Ciona, mouse and humans, we did not test the mouse Lama enhancer for activity in mouse, nor did we functionally interrogate the importance of the 12 bp spacing within this enhancer in the context of \textit{Ciona} or mouse. Conducting these studies would deepen our understanding of the conservation of grammar across chordates. We also identified a common logic of Zic, ETS, FoxA and Bra within Bra enhancers. While we know that deletion of the mouse Bra TNE enhancer does lead to loss of notochord in mouse, it would strengthen the study to manipulate the Zic, ETS, FoxA, Bra sites within the context of the mouse and zebrafish Bra/T enhancers to determine if the conservation of this logic is important for regulation of Bra.

%%%%%%%%%%%%%%%%%%%%%%%%%%%%%%%%%%%%%%%%%%%%%%%%%%%%%%%%%%%%%%%%%%%%%%%%%%%%%%%%
\section{STAR*Methods}
%%%%%%%%%%%%%%%%%%%%%%%%%%%%%%%%%%%%%%%%%%%%%%%%%%%%%%%%%%%%%%%%%%%%%%%%%%%%%%%%

%%% %%% %%% %%% %%% %%% %%% %%% %%% %%% %%% %%% %%% %%% %%% %%% %%% %%% %%% %%%
\subsection{Key resources table}

\begin{landscape} % this table is long, so it'll be multi-page landscape
    \begin{longtable}{p{.49\textwidth} p{.35\textwidth} p{.5\textwidth}}
        % Define the table title in the table of contents
        \caption{Key resources table} \\ \hline 

        % Define the table columns for the first and all subsequent pages
        \multicolumn{1}{l}{\textbf{REAGENT or RESOURCE}} & \multicolumn{1}{l}{\textbf{SOURCE}} & \multicolumn{1}{l}{\textbf{IDENTIFIER}} \\ \hline \endfirsthead

        \multicolumn{3}{l}%
        {{\textbf{\tablename\ \thetable{}.} Key resources table, \textit{continued from previous page}}} \\
        \hline 
        \multicolumn{1}{l}{\textbf{REAGENT or RESOURCE}} & \multicolumn{1}{l}{\textbf{SOURCE}} & \multicolumn{1}{l}{\textbf{IDENTIFIER}} \\ \hline\hline \endhead

        % Define the table footer for the first and all subsequent pages
        \hline \multicolumn{3}{r}{\textit{Continued on next page}} \\ \hline \endfoot
        \hline \endlastfoot
        
        % Start table content

        %%%%% %%%%% %%%%% %%%%% %%%%% %%%%% %%%%% %%%%% %%%%% %%%%% %%%%% %%%%%
        \textit{Deposited Data} \\ \hline
        snATAC-seq mouse E8.25 & Pijuan-Sala et al., 2020\cite{pijuan-sala2020} & GEO: \href{https://www.ncbi.nlm.nih.gov/geo/query/acc.cgi?acc=GSE133244}{GSE133244} \\ 
        FACS-sorted notochord RNA-Seq & Reeves et al., 2017\cite{reeves2017} & N/A \\
        Human reference genome NCBI build 38 & Genome Reference Consortium & \href{https://www.ncbi.nlm.nih.gov/grc/human}{NCBI, Human GRCh38 Reference} \\
        Mouse reference genome NCBI build 39 & Genome Reference Consortium & \href{https://www.ncbi.nlm.nih.gov/grc/mouse}{NCBI, Mouse GRCm39 Reference} \\
        \textit{Ciona} robusta genome & Satoh et al., 2005\cite{satou2005} & \href{http://ghost.zool.kyoto-u.ac.jp/cgi-bin/gb2/gbrowse/kh/}{Ghost Database} \\
        mouse ETS-1 universal PBM data & Wei et al., 2010\cite{wei2010} & \href{https://thebrain.bwh.harvard.edu/uniprobe/index.php}{UniProbe Database} \\
        ZEE library screen & This paper & N/A \\
        
        %%%%% %%%%% %%%%% %%%%% %%%%% %%%%% %%%%% %%%%% %%%%% %%%%% %%%%% %%%%%
        \hline \textit{Experimental Models: Organisms/Strains} \\ \hline
        \textit{Ciona} intestinalis type A (\textit{Ciona} robusta) & M-Rep & N/A \\

        %%%%% %%%%% %%%%% %%%%% %%%%% %%%%% %%%%% %%%%% %%%%% %%%%% %%%%% %%%%%
        \hline \textit{Oligonucleotides} \\ \hline
        Oligonucleotides for library screen, see Table S1 & This paper & N/A \\
        Oligonucleotides for mutagenesis, see Table S4 & This paper & N/A \\

        %%%%% %%%%% %%%%% %%%%% %%%%% %%%%% %%%%% %%%%% %%%%% %%%%% %%%%% %%%%%
        \hline \textit{Recombinant DNA} \\ \hline
        Plasmid: BraS bpFog$>$GFP & Farley Lab & N/A \\ 
        Plasmid: BraS -ZEE bpFog$>$GFP & This paper & N/A \\
        Plasmid: BraS rZE bpFog$>$GFP & This paper & N/A \\
        Plasmid: BraS -FoxA bpFog$>$GFP & This paper & N/A \\
        Plasmid: BraS -Bra bpFog$>$GFP & This paper & N/A \\ 
        Plasmid: BraS rZEFB bpFog$>$GFP & This paper & N/A \\
        Plasmid: Lama1 bpFog$>$GFP & This paper & N/A \\
        Plasmid: Lama1 bpFog$>$GFP & This paper & N/A \\
        Plasmid: Lama1 -E3 bpFog$>$GFP & This paper & N/A \\
        Plasmid: Lama1 -Z bpFog$>$GFP & This paper & N/A \\
        Plasmid: Lama1 RE3 bpFog$>$GFP & This paper & N/A \\

        %%%%% %%%%% %%%%% %%%%% %%%%% %%%%% %%%%% %%%%% %%%%% %%%%% %%%%% %%%%%
        \hline \textit{Software and Algorithms} \\ \hline
        Python (version 3.8.6)  & Python Software Foundation & \href{https://www.python.org}{https://www.python.org} \\
        Conda (version 4.9.2) & Anaconda, Inc. & \href{https://docs.conda.io/}{https://docs.conda.io} \\
        Bioconda  & Grüning et al., 2018 & \href{https://bioconda.github.io}{https://bioconda.github.io} \\
        Biopython (version 1.78) & Cock et al., 2009 & \href{https://biopython.org}{https://biopython.org} \\
        FastQC (version 0.11.9)	& Babraham Institute & \href{https://www.bioinformatics.babraham.ac.uk/}{https://www.bioinformatics.babraham.ac.uk} \\
        MultiQC (version 1.8) & Ewels et al., 2016 & \href{https://multiqc.info}{https://multiqc.info} \\
        FLASH (version 1.2.11) & Magoč et al., 2011 & \href{http://www.cbcb.umd.edu/software/flash}{http://www.cbcb.umd.edu/software/flash} \\
        \verb|pandas| (version 1.2.1) & NumFOCUS & \href{https://pandas.pydata.org}{https://pandas.pydata.org} \\
        \verb|numpy| (version 1.20.3) & Harris et al., 2020 & \href{https://numpy.org}{https://numpy.org} \\
        \verb|matplotlib| (version 3.2.2) & Hunter, 2007 & \href{https://matplotlib.org/}{https://matplotlib.org} \\
        \verb|scikit-learn| (version 0.24.1) & Pedregosa et al., 2011 & \href{https://scikit-learn.org/}{https://scikit-learn.org} \\
        \verb|seaborn| (version 0.11.1) & Waskom et al., 2021 & \href{https://seaborn.pydata.org/}{https://seaborn.pydata.org} \\
        \verb|Diverse-Logics-Notochord-Study| & Code used in this paper & \href{https://github.com/mragsac/Diverse-Logics-Notochord-Study}{Diverse-Logics-Notochord-Study GitHub} \\
    \end{longtable}
\end{landscape}

%%% %%% %%% %%% %%% %%% %%% %%% %%% %%% %%% %%% %%% %%% %%% %%% %%% %%% %%% %%%
\subsection{Resource availability}

%%% %%% %%% %%% %%% %%% %%% %%% %%% %%% %%% %%% %%% %%% %%% %%% %%% %%% %%% %%%
\subsubsection{Lead contact} 
Further information and requests for resources and reagents should be directed to and will be fulfilled by the lead contact, Emma K. Farley (\href{mailto:efarley@ucsd.edu}{efarley@ucsd.edu}). 

%%% %%% %%% %%% %%% %%% %%% %%% %%% %%% %%% %%% %%% %%% %%% %%% %%% %%% %%% %%%
\subsubsection{Materials availability} 
Plasmids generated in this study are available upon request. 

%%% %%% %%% %%% %%% %%% %%% %%% %%% %%% %%% %%% %%% %%% %%% %%% %%% %%% %%% %%%
\subsection{Experimental model and subject details}
\subsubsection{Tunicates}
Adult \textit{\textit{Ciona} intestinalis type A}, also known as \textit{\textit{Ciona} robusta}, were obtained from M-Rep and were maintained under constant illumination in seawater (obtained from Reliant Aquariums) at $18^\circ$C. \textit{Ciona} are hermaphroditic, therefore there is only one possible sex for individuals. Age or developmental stage of the embryos studied are indicated in the main text. 

%%% %%% %%% %%% %%% %%% %%% %%% %%% %%% %%% %%% %%% %%% %%% %%% %%% %%% %%% %%%
\subsection{Method details}
\subsubsection{Library Construction}
The genomic regions were ordered from Agilent Technologies with adapters containing BseRI sites. This was cloned into the custom-designed SEL-Seq (Synthetic Enhancer Library-Sequencing) vector using type II restriction enzyme BseRI. After cloning, the library was transformed into bacteria (MegaX DHB10 electrocompetent cells), and the culture was grown up until an OD of 1 was reached. DNA was extracted using the Macherey-Nagel Nucleobond Xtra Midi kit. A 30 bp barcode with adapters containing Esp3I sites was cloned into this library using type II restriction enzyme Esp3I. The library was transformed into bacteria (MegaX DHB10 electrocompetent cells) and grown up until an OD of 2 was reached. The DNA library was extracted from the bacteria using the Macherey-Nagel Nucleobond Xtra Midi kit. 

\subsubsection{Electroporation}
Dechlorination, \textit{in vitro} fertilization, and electroporation were performed as described previously in Farley et al., 2016.

\subsubsection{GFP reporter assays}
70 $\mu$g DNA was resuspended in 100 $\mu$L water and added to 400 $\mu$L of 0.96 M D-mannitol. Typically for each electroporation, eggs and sperm were collected from 10 adults. Embryos were fixed at the appropriate developmental stage for 15 minutes in 3.7\% formaldehyde. The tissue was then cleared in a series of washes of 0.3\% Triton-X in PBS and then of 0.01\% Triton-X in PBS. Samples were mounted in Prolong Gold. GFP images were obtained with an Olympus FV3000, using the 40X objective. All constructs were electroporated in three biological replicates.

\subsubsection{ZEE MPRA screen}
50 $\mu$g of the ZEE library was electroporated into ~5,000 fertilized eggs. Embryos developed until 5 hours and 30 minutes at $22^\circ$C. Embryos put into TriZol, and RNA was extracted following the manufacturer's instructions (Life Technologies). The RNA was DNase treated using Turbo DNaseI from Ambion following standard instructions. Poly-A selection was used to obtain only mRNA using poly-A biotinylated beads as per instructions (Dyna-beads, Life technologies). The mRNA was used in an RT reaction that was specifically selected for the barcoded mRNA (Transcriptor High Fidelity, Roche). The RT product was PCR amplified and size selected using Agencourt AMPure beads (Beckman Coulter), then checked for quality and size on the 2100 Bioanalyzer (Agilent) and sent for sequencing on the NovaSeq S4 PE100 mode (Illumina). Three biological replicates were sent for sequencing. 

The DNA was extracted by mixing the phenol-chloroform and interphase of TriZol extraction with 500 $\mu$L of Back Extraction Buffer (4 M guanidine thiocyanate, 50 mM sodium citrate, and 1 M Tris-base). DNA was treated with RnaseA (Thermo Fisher). DNA was cleaned up with phenol:chloroform:isoamyl alcohol (25:24:1) (Life Technologies). The DNA was PCR amplified and size selected using Agencourt AMPure beads (Beckman Coulter), then checked for quality and size on the 2100 Bioanalyzer (Agilent) and sent for sequencing on the NovaSeq S4 PE100 mode (Illumina). Three biological replicates were sent for sequencing.

\subsubsection{Counting Embryos}
For each experiment, once embryos had been mounted on slides, slide labels were covered with thick tape and randomly numbered by a laboratory member not involved in this project. Expression of GFP within embryos on each slides was counted blind. In each experiment, all comparative constructs were present, along with a slide with BraS as a reference. The X-Cite was turned on for 1hr before analysis to ensure the illumination intensity was constant. To determine levels of expression, high expression was set as visible with less than 25\% power on X-Cite illuminator. Fifty embryos were counted for each biological replicate.

\subsubsection{Acquisition of Images}
For enhancers being compared, images were taken from electroporations performed on the same day using identical settings. For representative images, embryos were chosen that represented the average from counting data. All images are subsequently cropped to an appropriate size. In each figure, the same exposure time for each image is shown to allow direct comparison.

\subsubsection{Identification of Putative Notochord Enhancers}
We developed a script that allows for the input of any organism’s genome in the fasta file format. The script first looks for an exact match of one of seven canonical Zic family binding sites and their reverse complements. We used the following sites in our search: \verb|CAGCTGTG| (Zic1/2/3), \verb|CCGCAGT| (Zic7/3/1), \verb|CCGCAGTC| (Zic6), \verb|CCCGCTGTG| (Zic1), \verb|CCAGCTGTG| (Zic3), \verb|CCGCTGTG| (Zic2/ZicC), and \verb|CCCGCAGTC| (Zic5) as these have been identified as functional in previous studies (Matsumoto et al., 2007a; Yagi et al., 2004). Next, we drew a window of 30 bp from either end of the canonical Zic family binding site and determine if there are at least two Ets binding site cores (i.e., either \verb|GGAA| or \verb|GGAT| and their respective reverse complement sequences) present within the window. The location of all regions containing at least a single Zic family binding site and two Ets binding sites are saved as part of the genome search.

\subsubsection{Scoring Relative Affinities of Binding Sites}
We calculated the relative ETS binding affinity using the median signal intensity of the universal protein binding microarray (PBM) data for mouse Ets-1 proteins from the UniProbe database (\href{http://thebrain.bwh.harvard.edu/uniprobe/index.php}{http://thebrain.bwh.harvard.edu/uniprobe/index.php}) (Hume et al., 2015). Previous studies have shown that the specificity of ETS family members is highly conserved even from flies to humans (Nitta et al., 2015; Wei et al., 2010), and thus ETS-1 is a good proxy for binding affinity in \textit{Ciona} ETS-1 which has a conserved DNA binding domain (Farley et al., 2015). The relative affinity score represents the fractional binding of median signal intensities of the native 8-mer motifs compared to the optimal 8-mer motifs for optimal Ets, which we defined as the \verb|CCGGAAGT| motif and its corresponding reverse complement.  

\subsubsection{Enhancer to Barcode Assignment \& Dictionary Analysis}
We constructed a dictionary of unique barcode tag-enhancer pairs by not allowing for any mismatches in the ~68 bp enhancers in our library and by not allowing barcode tag-enhancer pairs to have a read count of fewer than 150 reads. Additionally, we required all barcode tags to be 29 bp or 30 bp in length. If more than one barcode tag was associated with a single enhancer, we included all associated barcode tags that met the aforementioned barcode length and read count requirements. Within our dictionary, we did not find barcode tags that were matched to multiple enhancers. In total, the dictionary contains 90 enhancers that were uniquely mapped to one or more barcode tags, and a total of 640 barcode tag-enhancer pairs.

\subsubsection{SEL-Seq Data Analysis}
For the whole embryo library, we sequenced barcode tags from the DNA and RNA libraries on the Illumina HiSeq 4000. Reads that perfectly matched barcode tags in our barcode tag-enhancer dictionary were included in the subsequent analysis. We extracted all of the read sequences from the sequencing libraries and collapse them based on unique sequences, tabulating the number of times a unique sequence appears in the library. Next, we perform preliminary filtering on the unique sequences, filtering out sequences that (i) have \verb|N|’s present, (ii) are missing the GFP sequence after our expected location of the barcode tag, (iii) contain a barcode that is not an exact match to our enhancer-barcode tag dictionary, (iv) did not meet the minimum read cutoff of 25 reads. For the preliminary filtering step, all DNA and RNA libraries were processed separately. 

We normalize our data into RPM.  We filter our data to only include the set of barcode tags and enhancers that appear in DNA across all replicates and consolidate the expression for each enhancer by taking the average RPM value across barcode tags. For determining if an enhancer was active, we calculated an “enhancer activity score.” This score is calculated by averaging the $log_2({\frac{RNA}{DNA}})$ value across a given enhancer’s biological replicates. 

%%% %%% %%% %%% %%% %%% %%% %%% %%% %%% %%% %%% %%% %%% %%% %%% %%% %%% %%% %%%
\subsection{Quantification and statistical analysis}

To assess statistical differences between enhancer expression, Fischer’s exact test was used with the \verb|fisher.test| function in the R programming language. To assess statistical differences between enhancer expression levels, chi-squared test was used with the \verb|CHISQ.TEST| function in Microsoft Excel.

%%%%%%%%%%%%%%%%%%%%%%%%%%%%%%%%%%%%%%%%%%%%%%%%%%%%%%%%%%%%%%%%%%%%%%%%%%%%%%%%
\section{Data and code availability}
%%%%%%%%%%%%%%%%%%%%%%%%%%%%%%%%%%%%%%%%%%%%%%%%%%%%%%%%%%%%%%%%%%%%%%%%%%%%%%%%

Microscopy and scoring data reported in this paper will be shared by the lead contact upon request.

All ZEE screen sequencing data will be deposited to GEO and will be made publicly available as of the date of publication. The data will also be on SRA listed under the submission identifier \href{https://www.ncbi.nlm.nih.gov/sra/PRJNA861319}{PRJNA861319} and will be made available as of the date of publication. DOIs will be listed in the key resources table upon publication. 

All original code has been deposited to GitHub (\href{https://github.com/mragsac/Diverse-Logics-Notochord-Study}{https://github.com/mragsac/Diverse-Logics-Notochord-Study}) and is publicly available. DOIs will be listed in the key resources table upon publication. 

Any additional information required to reanalyze the data reported in this paper is available from the lead contact upon request. 

%%%%%%%%%%%%%%%%%%%%%%%%%%%%%%%%%%%%%%%%%%%%%%%%%%%%%%%%%%%%%%%%%%%%%%%%%%%%%%%%
\section{Acknowledgments}
%%%%%%%%%%%%%%%%%%%%%%%%%%%%%%%%%%%%%%%%%%%%%%%%%%%%%%%%%%%%%%%%%%%%%%%%%%%%%%%%

We thank the Farley Lab and Dennis Schifferl for helpful discussions. We thank Janet H.T. Song for her critical reading of the manuscript. We thank the UCSD IGM Genomics Center for their assistance with sequencing. B.P.S. was supported by NIH T32 GM133351. M.F.R. is supported by T32 GM008666. K.T. is supported by NSF 2109907 and 3DP2HG010013-01S1. G.A.J. was supported by a Hartwell Fellowship and NIH T32HL007444. E.K.F., B.P.S., M.F.R., K.T., G.A.J., J.L.G., S.H.L. were supported by NIH DP2HG010013.

% Include the acknowledgement that this is a reformatted reprint
Chapter 1,  in full,  is  a  reformatted  reprint  of  the  material as it appears in “Diverse logics encode notochord enhancers.”  Benjamin P. Song, Michelle F. Ragsac, Krissie Tellez, Granton A. Jindal, Jessica L. Grudzien, Sophia H. Le, Emma K. Farley. \textit{In Submission}, 2022.  The dissertation author was the primary investigator and co-first author of this paper.

%%% %%% %%% %%% %%% %%% %%% %%% %%% %%% %%% %%% %%% %%% %%% %%% %%% %%% %%% %%%
\subsection{Author contributions}
E.K.F., B.P.S., M.F.R, K.T., G.A.J. designed experiments. B.P.S., K.T., J.L.G., S.H.L. conducted experiments. M.F.R. conducted bioinformatic analyses. E.K.F. and B.P.S wrote the manuscript. All authors were involved in editing the manuscript.

%%% %%% %%% %%% %%% %%% %%% %%% %%% %%% %%% %%% %%% %%% %%% %%% %%% %%% %%% %%%
\subsection{Declaration of interests}
The authors declare no competing interests.

\chapter{Enhancers with similar binding factors express a wide range of activity}
\chapter{Future directions for enhancer grammar research}
\chapter{Generating open educational resources for university-level bioinformatics courses}
\label{chap:Bioinformatics education}

Rapid advances in next-generation sequencing (NGS) technologies have improved accessibility for experimentalists to generate genomic data at scale, but the barrier to entry to learning the computational skills necessary to analyze these datasets remains high. Despite computational courses being slowly integrated into the classical undergraduate Biology curricula, the breadth of scientific and technical knowledge needed to succeed in bioinformatics courses renders them inaccessible to individuals with incomplete foundations.

For many bioinformatics graduate programs, there can be an expectation for trainees to already have a baseline knowledge of programming and bioinformatics pipeline development. Inevitably, there is usually a proportion of admitted students that are non-computational. Not addressing this knowledge gap amongst non-computational scientists contributes to issues with student retention and morale within the program, especially for students of minoritized backgrounds. To directly address this need, I made it my mission in graduate school to develop inclusive teaching strategies in academically diverse classrooms to provide students with the skills necessary to confidently perform and understand bioinformatics analysis. Additionally, I advocated for and succeeded in making expectations of incoming bioinformatics graduate students clearer to improve the retention of trainees. As a consequence of the quarantine in response to the global SARS-CoV-2 pandemic, I taught in-person and fully online modalities. 

%%%%%%%%%%%%%%%%%%%%%%%%%%%%%%%%%%%%%%%%%%%%%%%%%%%%%%%%%%%%%%%%%%%%%%%%%%%%%%%%
\section{Introduction}
%%%%%%%%%%%%%%%%%%%%%%%%%%%%%%%%%%%%%%%%%%%%%%%%%%%%%%%%%%%%%%%%%%%%%%%%%%%%%%%%

\subsection{Bioinformatics as a specialized data science discipline}

Massively parallel or next-generation sequencing (NGS) provides researchers with an exceedingly flexible set of molecular techniques to study various types of biological sequence data at a large scale. Currently, most sequencing is performed in research laboratories that need sustainable strategies for handling computational processing and data storage \cite{barone2017,leonelli2019,marx2013,pal2020,stephens2015}. Ergo, it has become necessary for biological and biomedical scientists at all educational levels to have some basic computational education for successful research \cite{attwood2019,pevzner2009a,rubinstein2014,tan2009}. 

The computational field of data science is an interdisciplinary discipline that utilizes algorithms, statistics, and scientific methods to extrapolate knowledge from various data types \cite{attwood2019,cao2017,getoor2019,stephens2015}. Thus as an amalgamation of disciplines, bioinformatics can be considered a subset of data science as it requires the ability to integrate concepts across biology, mathematics, computer science, and statistics. Additionally, bioinformatics requires substantial subfield-specific knowledge about particular computational tools and the biological context in which data was generated to generate accurate interpretations of data \cite{attwood2019,leonelli2019,marx2013,pal2020,rubinstein2014,tan2009}. Therefore, universities should consider this necessary breadth of knowledge in designing new undergraduate Biology curricula for their students to apply computational skills appropriately. 

While universities have started integrating computational modules into undergraduate and graduate biology student training, these modifications have not happened consistently across programs. Additionally, integrating computational coursework into current programs does not address the learning gap for scientists that wish to learn bioinformatics later in their careers when they do not have access to a classroom \cite{attwood2019,barone2017,zhan2019}. These learners often seek out opportunities to take computational courses in their own time, including in the university setting where many work. Unfortunately, the rising interest in computer science has imposed course enrollment caps in introductory programming and algorithms undergraduate courses due to the unmatched supply of available classes and instructors \cite{brodley2022,nager2016,shein2019,camp2015,jaggars2016}. These enrollment caps then severely limit the opportunities for non-undergraduate individuals to supplement their professional experience with basic computational skills in an academic setting \cite{jaggars2016,brodley2022,nager2016,camp2015,backofen2006}. An added issue is that for the few fortunate enough to enroll in a pure computer science course and learn the computational problem-solving mindset, these courses can then be difficult to translate directly into running bioinformatics pipelines. Thus, there is a need to develop accessible university-level bioinformatics course material.

\subsection{Placing bioinformatics in the context of discipline-based education research}

Educational researchers focus on scientific investigation of topics within the field of education to improve teaching and learning practices \cite{charles1998,nationalresearchcouncilu.s.2005}. While educational researchers can focus on general teaching topics, such as effective teaching methods for learners of various ages, discipline-based education research (DBER) evaluates learning and teaching in a particular discipline, such as biology or computer science \cite{slater2015}. 

While DBER looks at different disciplines separately, there are concrete similarities in their multidisciplinary nature and overall goal of improving the learning experience for students at the primary, secondary, and higher education levels within their respective fields. Computer science education or computing education research addresses learning and teaching in computer science \cite{randolph2008,almstrum2005,cooper2014,pears2005,malmi2010,guzdial2015}. For this DBER field, the Association for Computing Machinery runs a special interest group (SIG) on computer science education (CSE) research known as SIGCSE \footnote{\href{https://www.sigcse.org/}{https://www.sigcse.org/}}, whose affiliated conferences are some of the top venues for educational scholars to discuss topics related to computing and teaching methods. Computing education research covers an array of questions, including studying the retention of students, the difficulties of novice programmers, and the effectiveness of learning tools employed in the classroom \cite{randolph2008,almstrum2005,cooper2014,pears2005,malmi2010,guzdial2015}. Similarly, biology education research concerns the promotion and accessibility of biology education within the classroom and teaching laboratory settings \cite{bahar1999,mintzes2001,labov2010,pranjol2022,brownell2015,heim2019,bakshi2016}. Many biology education research programs also evaluate the effectiveness of course-based undergraduate research experiences (CUREs) in increasing interest in science and providing a proper intervention to encourage higher representation of historically marginalized students within academia \cite{pranjol2022,brownell2015,heim2019,bakshi2016}. 

As a highly multidisciplinary field, bioinformatics presents a rare opportunity to understand how students learn and synthesize information spanning disparate fields and how teaching pedagogies unique to particular disciplines can be effective or ineffective. Students wishing to learn bioinformatics come from different backgrounds. Additionally, the suggested core competencies for bioinformatics also differ depending on the professional level of the individual and desired skill set for the role they are in \cite{tan2009,rubinstein2014,attwood2019,zhan2019,labov2010,mulder2018}. Teaching methods should then differ in how they approach students with a limited programming background, students with limited molecular biology knowledge, or students with experience in both fields separately but not integrated.  

%%%%%%%%%%%%%%%%%%%%%%%%%%%%%%%%%%%%%%%%%%%%%%%%%%%%%%%%%%%%%%%%%%%%%%%%%%%%%%%%
\section{Methods}
%%%%%%%%%%%%%%%%%%%%%%%%%%%%%%%%%%%%%%%%%%%%%%%%%%%%%%%%%%%%%%%%%%%%%%%%%%%%%%%%

\subsection{Graduate bioinformatics training at the University of California, San Diego}

The University of California, San Diego (UCSD) offers bioinformatics training at the undergraduate \footnote{Students are able to get undergraduate degrees in bioinformatics from one of three departments: the Department of Bioengineering, the Department of Biology, or the Department of Computer Science and Engineering. The course requirements vary slightly depending on the department.} and graduate degree \footnote{\href{https://bioinformatics.ucsd.edu/}{\texttt{https://bioinformatics.ucsd.edu/}}} levels and the professional certification level at the university's extension learning center \footnote{\href{https://extendedstudies.ucsd.edu/courses-and-programs/applied-bioinformatics}{\texttt{https://extendedstudies.ucsd.edu/courses-and-programs/applied-bioinformatics}}}. Within this chapter, I will focus on the introductory bioinformatics training that I provided to masters, doctoral, and professional students across the courses and modalities provided in Table \ref{tab:course-table}. 

\begin{small}
    \tolerance=1 
    \emergencystretch=\maxdimen 
    \hyphenpenalty=10000 
    \hbadness=10000
    \begin{landscape} % this table is long, so it'll be multi-page landscape
        \begin{table}[]
            \caption{Bioinformatics courses taught at the University of California, San Diego}
            \label{tab:course-table}
            \begin{tabular}{p{.15\textwidth} p{.25\textwidth} p{.42\textwidth} p{.17\textwidth} p{.23\textwidth}}
            \hline
            \textbf{QUARTER} & \textbf{DEPARTMENT} & \textbf{COURSE NAME} & \textbf{MODALITY} & \textbf{WEBSITE} \\ \hline\hline \\

            Spring Quarter 2019 (Apr-Jun) & Scripps Institute of Oceanography (SIO) & SIOB 242C: Marine Biotechnology III, Introduction to Bioinformatics & In-Person & \textit{N/A} \\ \\ \hline \\ 
            
            Winter Quarter 2020 (Jan-Mar) & School of Medicine, Department of Cellular and Molecular Medicine & CMM 262/BIOM 262: Quantitative Methods in Genetics & In-Person & \href{https://github.com/biom262/cmm262-2020}{\texttt{cmm262-2020}} GitHub Repository \\ \\ \hline \\ 
            
            September 2020 (Before Fall Quarter 2020) & Jacobs School of Engineering & Bioinformatics \& Systems Biology Program Bootcamp & Online & \href{https://github.com/mragsac/BISB-Bootcamp-2020}{\texttt{BISB-Bootcamp-2020}} GitHub Repository \\ \\ \hline \\  

            Winter Quarter 2021 (Jan-Mar) & School of Medicine, Department of Cellular and Molecular Medicine & CMM 262/BIOM 262: Quantitative Methods in Genetics & Online & \href{https://github.com/biom262/cmm262-2021}{\texttt{cmm262-2021}} GitHub Repository \\ \\ \hline \\ 
            
            September Quarter 2021 (Before Fall Quarter 2021) & Jacobs School of Engineering & Bioinformatics \& Systems Biology Program Bootcamp & Online & \href{https://github.com/mragsac/BISB-Bootcamp-2021}{\texttt{BISB-Bootcamp-2021}} GitHub Repository \\ \\ \hline
            
            \end{tabular}
        \end{table}
    \end{landscape}
\end{small}

\subsubsection{SIOB 242C: Marine Biotechnology III, Introduction to Bioinformatics}

Conceptualized and taught by Theresa (Terry) Gaasterland, Ph.D. from the Scripps Institute of Oceanography (SIO), SIOB 242C is designed to give students an introduction to using high-performance computing systems to analyze real, primary RNA-sequencing data using command-line tools. In this class, there is a lecture once a week involving file manipulation and genomic data regular expressions in Unix, along with an accompanying take-home homework assignment. For this course, I acted as the only teaching assistant and hosted a weekly problem-solving session and office hours on an as-needed basis. Due to the small size of the graduate program at SIO, there were only ten students formally enrolled in the class. Additionally, the majority of students enrolled in the course had a background in marine biology without much computational experience.

\subsubsection{CMM 262/BIOM 262: Quantitative Methods in Genetics}
CMM 262 (also cross-listed as BIOM 262) is a required course for the UCSD Genetics Training Program and is designed to teach experimental and analytical approaches in modern genetics and genomics in several topic areas. I taught CMM 262 in Winter Quarter 2020 and Winter Quarter 2021 alongside Alon Goren, Ph.D. from the UCSD School of Medicine, and three other graduate students from the BISB Program. In this class, a guest instructor specializing in a particular subtopic of genetics presents two lectures to a class of approximately fifty biomedical sciences students. The teaching assistants for CMM 262 were responsible for coordinating guest faulty speakers, managing the distribution of course materials, grading course assignments and exams, and holding office hours for students. In the 2021 iteration of the class, I served as one of the lead teaching assistants. Additionally, due to the SARS-CoV-2 pandemic, this course was taught in-person and hybrid for 2020 and entirely online for 2021. Across both years, the majority of students enrolled in CMM 262 had a background in biomedical sciences without much exposure to computer programming.

\subsubsection{Bioinformatics \& Systems Biology Program Bootcamp}
Held every year during the week before the start of the academic year, the BISB Bootcamp is a student-run training course for incoming students to the BISB Doctoral Program. Through the BISB Bootcamp, incoming students are exposed to faculty research within the program and given a primer on topics in molecular biology, genetics, statistics, machine learning, computer science, and professional development meant to prepare them for their time in graduate school. As one of the course instructors, I was responsible for disseminating course materials to students before they arrived at UCSD, designing academic instructional modules, and logistical planning of the course. Due to the SARS-CoV-2 pandemic, the BISB Bootcamp was taught entirely online for 2020 and 2021. The students admitted to the BISB program are academically diverse, thus, students had varying degrees of exposure to computer programming and molecular biology.

\subsection{Publication of locally delivered bioinformatics course materials as open educational resources}

Open education is an educational movement founded on accessibility, transparency, and collaboration. Open education aims to provide broader access to the learning and training provided through formal educational systems, such as the university environment \cite{weller2014,mishra2017,hylen,abri2018,colvard2018}. To provide greater access to educational materials to individuals in various time zones worldwide, open education programs typically take advantage of online platforms to distribute content, such as open educational resources (OERs). OERs are educational resources (e.g., course materials, textbooks, multimedia applications) in the public domain that are openly available for instructors or students to retain, reuse, revise, remix, or redistribute without an accompanying need to pay royalties or licensing fees \cite{islim2016,geith2008,moore2022,colvard2018,abri2018,hylen,mishra2017,weller2014}.  

Most course materials I developed for the bioinformatics courses I taught locally at UCSD were distributed as OERs through the GitHub platform (Table \ref{tab:course-table}) to support the open education paradigm. By distributing the materials through GitHub, I sought to increase the reach of the high-quality bioinformatics educational materials I created for UCSD while allowing people to revise, add, or remove course content as desired while using GitHub's version-control feature for transparency of modifications. One of the fundamental guiding principles of open education is that everyone worldwide should have access to high-quality educational experiences and resources. By publicizing the course content for CMM262 and the BISB Bootcamp, I aimed to eliminate barriers to this goal by reducing the high monetary costs of bioinformatics training and encouraging collaboration between scholars and educators in the field.

%%%%%%%%%%%%%%%%%%%%%%%%%%%%%%%%%%%%%%%%%%%%%%%%%%%%%%%%%%%%%%%%%%%%%%%%%%%%%%%%
\section{Results}
%%%%%%%%%%%%%%%%%%%%%%%%%%%%%%%%%%%%%%%%%%%%%%%%%%%%%%%%%%%%%%%%%%%%%%%%%%%%%%%%

Generally, bioinformatics courses often range in the course's duration and the scope of the material covered (i.e., lecturing on single versus multiple topics). One of the most common formats includes short courses that cover a particular topic or analysis pipeline (e.g., evaluating single-cell RNA-sequencing analysis, genome-wide association studies, etc.) \footnote{These are common at certain institutions and bioinformatics core facilities such as Cold Spring Harbor (\url{https://www.cshl.edu/meetings-courses-program/}), the University of California, Davis campus (\url{https://bioinformatics.ucdavis.edu/training}), the Jackson Laboratory (\url{https://www.jax.org/education-and-learning/course-and-conferences/bioinformatics-training-program}), and many others.}. During graduate school, I taught a total of five comprehensive Python, R, and UNIX-based bioinformatics courses that covered multiple analysis pipelines related to transcriptomics, epigenetics, and population genetics (Table \ref{tab:course-table}). Within this section, I will discuss my strategies as a member of the teaching team for these courses to cater to the needs of students.

\subsection{Incorporating practical computational modules into course design}

There are many free bioinformatics online tutorials in the form of blog posts, GitHub-stored Jupyter Notebooks, and RMarkdown Books. Unfortunately, biological and biomedical scientists sometimes find it difficult to directly apply these generic pipelines to their data, especially when they lack programming knowledge or the computational resources needed to run a particular analysis. With any programming language, students require baseline skills in learning how to decode runtime errors and how to resolve these errors. The added complexity of data analytics requires that students analyzing biological data understand how the parameters for the tools they use impact their overall analysis and how these parameters balance with the system their study is conducted in. Thus, it can be difficult for students lacking the programming skills or theoretical biology background to apply off-the-shelf bioinformatics tools appropriately without guidance. 

Showcasing practical examples was important in ensuring students understood how to apply bioinformatics pipelines appropriately and in tempering expectations for bioinformatics as a whole. For example, when surveyed about a highly-interactive lecture on genome-wide association studies (GWAS) for CMM 262 taught in Winter 2021, students had high praise for the guest instructor: one student remarked in the free response section of the survey, “\textit{I really liked the coding exercises and doing them in real-time, it made me think through what was going on in the data…}” and another student mentioned, “\textit{The best part of the} [lecture] \textit{was the fact that the lines were not already filled so the class was a little bit more active…}” To foster students' feelings of being active participants in lectures, we encouraged lecturers for CMM 262 to incorporate live programming in their lectures. Additionally, to ensure that students from SIOB 242C, CMM 242, and the BISB Bootcamp could apply knowledge from the courses to data produced from their present and future research labs, we specifically showcased well-known, existing community tools. Examples include \verb|samtools| \cite{danecek2021}, \verb|STAR| \cite{dobin2013}, \verb|seurat| \cite{hao2021}, \verb|scanpy| \cite{wolf2018}, \verb|MACS2| \cite{zhang2008}, and others. This ensured that after finishing our class, students would have access to a wealth of community resources and online forums with potential answers to their questions or answers to particular error prompts.

\subsection{Comparison of delivery methods for deploying bioinformatics assignments}

Many academic laboratories use high-performance computing (HPC) or cloud-based systems to analyze biological and biomedical datasets that cannot easily be processed on a laptop or desktop computer \cite{pal2020,marx2013,stephens2015,leonelli2019}. One example is the UCSD Triton Shared Compute Cluster (TSCC) \footnote{\href{https://sdsc.edu/services/hpc/tscc/index.html}{https://sdsc.edu/services/hpc/tscc/index.html}} housed at the San Diego Supercomputer Center (SDSC) \footnote{\href{https://sdsc.edu/}{https://sdsc.edu/}}. TSCC is a condo cluster program that researchers can buy into through hardware purchases of computing nodes or by purchasing computing hours as account credits. Despite the commonality of using Jupyter Notebooks for data exploration and visualization, academic labs can differ in how to access HPC or cloud-based computing systems based on ease of access, monetary constraints, or firewall requirements (i.e., medical data files protected by HIPAA have particular security requirements), monetary constraints, and ease of access. Two common methods to access Jupyter Notebooks include command line-based and on-demand-based methods. However, both methods provide pros and cons for first-time bioinformatics learners. 

As part of SIOB 242C, CMM 242 taught in Winter Quarter 2020, and the BISB Bootcamp taught in September 2020, we worked with the San Diego Supercomputer Center (SDSC) to provide training accounts with enough credits for the entire quarter. Additionally, for the ChIP-sequencing analysis module taught in CMM 262 in Winter Quarter 2021, students were encouraged to complete bioinformatics pipelines similar to how you would on TSCC. With TSCC, students could learn additional skills in navigating the UNIX command line and using job scheduling systems to submit computational tasks. Students also learned how to customize software environments (e.g., conda environments) to cater to particular analysis pipelines. However, this additional layer between the student and course assignments introduced a larger learning curve toward the beginning of the course, especially for those that lacked prior programming experience. When students in CMM 262 taught in Winter 2021 were surveyed regarding their experiences learning to analyze ChIP-sequencing data through hands-on UNIX commands, many students felt the module was presented clearly. Upon being asked, “\textit{Did the lecturer present material clearly and understandably?}”, 33.3\% of students indicated Strongly Agree (9/27), 37\% indicated Agree (10/27), 18.5\% indicated Neither Agree nor Disagree (5/27), 11.1\% indicated Disagreed (3/27), and indicated 0.0\% Strongly Disagreed (0/27). But when reviewing the free response section of the survey, some students felt “\textit{…it was easy to fall behind…}” or “\textit{…the speed was too fast…}” whereas others felt that the pace “\textit{…could have been faster.}” The dichotomy in feedback in the free response section reflected the vast differences in technical background students had and their ability to follow along in the module. 

In contrast to 2020, for CMM 262 taught in Winter Quarter 2021 and the BISB Bootcamp taught in September 2021, we primarily used the JupyterHub platform\footnote{\url{https://jupyter.org/hub}} to easily deploy data science notebooks to students that shared markdown text of lesson material alongside code blocks using bioinformatics tools. With UCSD's JupyterHub platform, DataHub\footnote{\url{https://datahub.ucsd.edu/}}, students could immediately jump into a particular course exercise without worrying about package installations or data transfers-these were all aspects handled by the teaching staff and UCSD Educational Teaching Services. Unfortunately, upon coming out of the class, these students can face a larger barrier to pursuing bioinformatics in their research as they may have an idea of how to apply analysis platforms from their coursework but not how to set up or access the bioinformatics infrastructure they need. For example, when surveyed regarding the utility of Jupyter Notebooks in CMM 262 during Winter 2021, one student commented, “\textit{…since the code-along is being done on Jupyter Hub, I would like some additional resources or links on how to set up my machine (PC) for coding outside of the Hub.}” Additionally, several students remarked during office hours that a Jupyter Notebook-only approach without much practical, hands-on learning was less engaging. In a final course survey, there were responses such as, “\textit{…learning with just the} [Jupyter] \textit{notebook feels passive…}” and “\textit{…} [if] \textit{we had just executed} [ChIP-sequencing commands] \textit{in the notebook, I wouldn't have understood it as well, although I'm glad to have the notebook as a reference.}”

\subsection{Unifying students across diverse academic backgrounds in the classroom}

Designing courses for students from different backgrounds can be extremely challenging regardless of the subject taught. Comprehensive introductory bioinformatics courses are no exception: the variation in course topics and the wide array of student academic backgrounds from typically non-intermixing fields make it difficult to design a course that can unify rather than alienate students in the classroom. Students entering bioinformatics courses cover various specialties, from biological and biomedical sciences to the physical and computational sciences. One of the largest challenges in designing a comprehensive bioinformatics course is to develop in-class exercises and lectures that can unify the classroom rather than unintentionally isolate groups of students based on their knowledge gaps. Typical knowledge gaps include programming, molecular biology, lab experience, and statistics. 

To cater to the diverse needs of students, my main goal was to enforce a culture of inclusivity of all academic and socioeconomic backgrounds to foster a less intimidating and safer classroom environment. Between SIOB 242C, CMM 262, and the BISB Bootcamp, there are distinct differences between the backgrounds of students. For example, students in SIOB 242C and CMM 262 have a primarily experimental biology background and often have limited programming exposure. On the other hand, students in the BISB Bootcamp have a highly diverse population of students that are often more computational without having extensive experience in genetics, genomics, and molecular biology techniques. For these two groups, the teaching style varies to accommodate their unique backgrounds. 

\subsection{Teaching students with biological backgrounds to adopt a growth mindset in learning bioinformatics}

In teaching students that have limited computational experience, my initial goal is to make computer science and computing more accessible and less intimidating, especially for groups of students historically excluded from these subjects. Due to inequitable access to computer science education before college, many students can feel unprepared for or unsuitable for introductory computer science coursework \cite{wang2016,stout2017,brown2010,kirby1990,hubbardcheuoua2021,lewis2015,rogers2016,shah2014,shinohara2020}. Psychological roadblocks-such as stereotype threat and imposter syndrome-can also contribute to students' perceived potential success in computer science \cite{steele2002,thoman2013,stout2017,smith2008,eschenbach2014,bell2003,kumar2012,halljr.2018,falkner2015,rosenstein2020}. When I teach introductory bioinformatics courses, I address these concerns to boost students' confidence and to foster a classroom environment where these concerns can be discussed openly with other students and the teaching staff. Additionally, I explicitly state that prior programming experience is not required to succeed within introductory bioinformatics courses to eliminate preconceived notions about required background knowledge before instruction takes place.

Stereotype threat is when an individual feels at risk of confirming negative stereotypes about the group of which they are a member \cite{steele2002,thoman2013,smith2008,eschenbach2014,bell2003}. Situational factors that contribute to stereotype threat include the task's difficulty at hand, the belief that the task measures their abilities, and the relevance of the stereotype to the task. Stereotype threat is believed to be a psychological barrier to students' engagement in computer science due to its ability to contribute to diminished confidence, poor performance, and loss of interest in the field, especially for minoritized students \cite{steele2002,smith2008,eschenbach2014,bell2003,kumar2012,halljr.2018,falkner2015}. While computer science courses tend to attract more men and more white and South Asian or East Asian students, biological science courses comparatively attract more women and more Latine and Black students \cite{nationalcenterforscienceandengineeringstatisticsncses2019,falkner2015,halljr.2018,bell2003}. Thus, as an instructor, I try to be welcoming and compassionate towards the women and non-binary students and students from minoritized groups that enter the classroom. In teaching bioinformatics, I address students' concerns as they arise to ensure retention and foster interest in the field.

One of the main points of concern I observed from students in SIOB 242C and CMM 262 was their inability to learn how to program late in their academic careers. Thus, the primary strategy I employed to help these students was to encourage the adoption of a growth mindset. One theory of intelligence holds that people can be categorized into two groups based on their implicit beliefs about their ability to learn. People with a fixed mindset believe that learning ability is innate, whereas people with a growth mindset believe knowledge can be acquired through effort and studying \cite{hochanadel2015,rhew2018,morreale2021,stout2017}. Computer science is a difficult subject for first-time learners due to (i) the steep initial learning curve in learning a new language, (ii) the detail-oriented nature required to meet syntaxial requirements, and (iii) the constructive nature of computer science as a discipline \cite{rivers2016,parker2014,qian2017}. It is important to address each of these difficulties during instruction to encourage the development of a growth mindset in the classroom. 

Most of my teaching success was derived from live programming to solve bioinformatics problems during course instruction. Because of the steep initial learning curve involved in computer science, I feel that concepts should be introduced slowly and explained explicitly. Within SIOB 242C and CMM 262, I incorporated live programming in my teaching to naturally explain new computer science concepts (e.g., variable declaration, for and while loops, conditional expressions) as they pertained to solving a bioinformatics problem in real-time. In live programming, I aimed to demystify the black box that bioinformatics can often feel like and provide a practical example of how computational concepts can be easily applied to students' work outside the classroom to encourage engagement with the material. For example, when teaching a learning module on basic statistics for CMM 262 during Winter 2021, a common point of feedback was that students “\textit{…found the practical examples in notebooks extremely helpful.}” Several students also felt motivated to program on their own, and one comment indicated that “\textit{…as someone who is brand new to programming it might be nice if there were a few brief practice problems we could try out on our own and see posted answer keys later…}” Computer programming requires that people be meticulous about noticing syntaxial errors in particular programming languages \cite{sentance2015,rivers2016}. Through live programming, I was also able to touch on the importance of being detail-oriented when it comes to computer programming. During live programming demos, students often pointed out errors in my programming and suggested modifications to my code to make things run successfully. Finally, we could trace through errors in programming logic together as a class, enforcing a unified community spirit in the classroom. 

\subsection{Reducing information overload in teaching bioinformatics to computational students}

Introductory biology courses typically cover a multitude of topics, and it is well known that students at the secondary and undergraduate levels face difficulties in learning biological concepts \cite{bahar1999,lazarowitz1992,kelly-laubscher2016,anderson2022}. For instance, biology classes have overloaded curricula and cover abstract topics \cite{bahar1999,lazarowitz1992,anderson2022}. These two factors combined often lead to preconceived notions of rote memorization being the defining feature of biology as a whole \cite{bahar1999,mcdaniel2022}. Within the BISB Bootcamp, there was a larger proportion of computational students compared to SIOB 242C and CMM 262. These students had engineering, physical, or computer science backgrounds but no extensive experience with molecular biology or genetics. My primary goal in teaching these students was to teach core biology concepts that present themselves in commonly-discussed bioinformatics problems to reduce cognitive overload. 

Cognitive overload or information overload occurs when you are exposed to more details than you can process at any given time. Additionally, cognitive overload can manifest as mental fatigue, reduced attention span, and behavioral changes \cite{koc-januchta2022,ehrmann2022,fox2007,kirsh2000a}. Similarly to teaching experimental biology students, I employed one strategy to reduce cognitive overload: slowly introducing biology terminology and concepts as they become relevant to the bioinformatics problem. This teaching method can also be seen through a widely-taken bioinformatics Coursera course series developed by Pavel Pevzner and Philip Compeau, as they introduce questions in biology that can use computational methods for answer generation\footnote{https://stepik.org/course/55789/}. 

Within the BISB Bootcamp, this teaching method was used when students were presented with the problem of looking for transcription factor binding sites within a sequence. In this example, students were introduced to several ideas in the following order: 

\par\noindent\dotfill

\begin{enumerate}
    \item \textbf{Providing Background Information on the Biological Problem}
    \begin{enumerate}
        \item Consider a sentence as a string of words made up of individual letters or characters. 
        \item Also, consider that genetic information is contained within all the cells of the body as DNA.
        \begin{enumerate}
            \item DNA consists of the nucleic acids adenine (\verb|A|), thymine (\verb|T|), cytosine (\verb|C|), and guanine (\verb|G|).
        \end{enumerate}
        \item DNA can be represented as a string of different characters representing the nucleic acids.  
        \item There are regions of DNA that transcription factor proteins can bind to through the recognition of short DNA sequences (e.g., \verb|GATA|) to activate particular genes.
        \begin{enumerate}
            \item These regions are otherwise known as transcription factor binding sites.
        \end{enumerate}
    \end{enumerate}
    \item \textbf{Defining the Biological Problem and the Associated Computational Problem}
    \begin{enumerate}
        \item If we consider DNA as a sentence, these binding regions can be the words in our sentence that we're trying to understand the meaning of (e.g., how come certain transcription factors activate certain genes, and are there any patterns?). 
        \item To further understand gene activation, we need to be able to recognize where transcription factor binding sites are within a DNA sequence!
        \item Looking for transcription factor binding sites within a DNA sequence can be considered a computational “search” problem of looking for a substring within a string! 
    \end{enumerate}
    \item \textbf{Developing a Bioinformatics Solution}
    \begin{enumerate}
        \item We can define variables that represent the DNA sequence and the binding site. 
        \item Next, loop through each position in the DNA sequence to see if the transcription factor binding site matches the start of the sequence. 
        \begin{enumerate}
            \item If the site is found, we've successfully identified the location of a binding site! 
            \begin{enumerate}
                \item We can save the position with a match and then continue to the next position in the sequence to look for more binding site matches. 
            \end{enumerate}
            \item If we have looked at all positions in the entire DNA sequence and haven't found a binding site match, then the site does not exist. 
        \end{enumerate}
    \end{enumerate}
\end{enumerate}

\par\noindent\dotfill

When prompted with an optional survey to score this module from a range of one for “\textit{Uninformative}” to five for “\textit{Transcendent},” 37.5\% of students (6/16) rated the module with a score of five, 6.3\% of students (1/16) rated the module with a score of four, 37.5\% of students (6/16) rated the module with a score of three, and 18.8\% of students (3/16) left the field blank. In the free response section to provide feedback for this module, students were relatively satisfied with the teaching format, commenting, “[the interactive module] \textit{was definitely needed for the rest of the week,}” “\textit{I think it was useful and I got more out of the session I attended than I would have from the other} [lecture without programming in biology],” and “\textit{It was very useful.}” Additionally, one student with a larger background in computer science commented, “\textit{…I was able to do the CS project/presentation} [with] \textit{no problem} [while interacting] \textit{with the biology live.}” 

While this particular example relies on prior molecular biology knowledge of DNA and proteins, we reduced the amount of background information required to understand the overall goal of this example and the applicable bioinformatics problem. In particular, we abstracted the concept of gene activation for the audience by not mentioning other parts of the system, such as transcriptional cofactors, enhancers, promoters, or genome methylation. Students were then able to easily recognize the value of computational methods when integrated with molecular biology, and our teaching methodology helped spark interest in students' interest in theoretical molecular biology across various topics. For example, one student provided the following comment, “\textit{As someone with no biology background, it did go a bit over my head. However, I still found it useful to hear about different techniques even if I didn't fully understand them … I had a great discussion with} [the Bootcamp instructors] \textit{at the end of the lecture about possibly working in a wet lab.}” 

\subsection{Using interactive teaching pedagogies to encourage student participation} 

The oldest teaching pedagogy is known to be “teacher-centric,” where the instructor lectures students who enter the classroom as a \textit{tabula rasa}, expected to passively receive the knowledge being disseminated. Under this paradigm, the instructor is the core regulator of knowledge in the classroom: they do most of the talking, set the rules and learning goals, and drive the direction of follow-up discussions \cite{freire2000}. Recently, classrooms-especially juvenile classrooms--have started adopting a “student-centric” pedagogy. In this environment, students control the direction of learning through collaborative discussions with their peers after being given the required conditions and tools by the instructor \cite{estes2004,wright2011,brush2000}. While teacher-centric and student-centric methods fall at opposite ends of the spectrum, instructors use varying proportions of each methodology, known as “interactive teaching” \cite{senthamarai2018,kennewell2008}.

Bioinformatics is based on technological advancements in biology and, thus, relies heavily on access to a computer, especially for data analytics. For bioinformatics courses focused on data analysis rather than algorithmic design, we can easily incorporate interactive teaching into course lectures. Course lectures were modified in real-time based on student feedback in SIOB 242C, CMM 262, and the BISB Bootcamp. Each concept was taught as a “block” consisting of four components: (i) a molecular biology concept (e.g., genome sequences), (ii) an open question concerning the concept presented (e.g., comparing genomes), (iii) a parallel computer science concept (e.g., string comparisons), and (iv) an example computational solution to the question (e.g., genome/string alignment with dynamic programming). By being upfront with the interdisciplinary nature of bioinformatics problems, students of all backgrounds were engaged in asking questions during course instruction and providing solutions to questions provided during live programming demonstrations. 

\subsection{The impact of COVID-19 on teaching university-level bioinformatics courses in 2020 and 2021}

In recent years, universities have adopted online educational tools into regular instruction to provide greater accessibility to course materials, external resources, and grading information online. For example, many universities use the CANVAS \footnote{\href{https://www.instructure.com/canvas}{https://www.instructure.com/canvas}} web-based learning management system (LMS) \cite{marachi2020,beldarrain2006,endozo2020}. Many engineering and computational courses use standardized platforms for student communication and course assessments, such as Piazza \footnote{\href{https://piazza.com/}{https://piazza.com/}} and GradeScope \footnote{\href{https://www.gradescope.com/}{https://www.gradescope.com/}}. In addition to LMS platforms, some universities have started exploring “flipped classroom” formats in which students encounter lecture material independently before dedicating all in-person instructional time to discussion-like sessions \cite{polat2022,long2017,jared2014}. However, towards the end of 2020, this gradual process of virtualizing traditional in-person courses was greatly accelerated by the high aerosol transmissibility of the SARS-CoV-2 virus \cite{jones2020,holingue2020,wu2020}. 

The emergency of the Coronavirus Disease 2019 (COVID-19) crisis forced instructors worldwide to translate their in-person courses into virtual environments, introducing difficulties in promoting interaction between students and instructors, especially in a medium unfamiliar to many. Despite bioinformatics' reliance on virtual resources and computers as a field, many still faced challenges translating successful in-person courses to online-only mediums. Due to COVID-19, I was forced to adapt the coursework for CMM 262 and the BISB Bootcamp on the fly (Table \ref{tab:course-table}). In addition to the commonly observed issue of student engagement, one of the largest challenges in CMM 262 and the BISB Bootcamp was losing important in-person interactions in teaching and learning programming for the first time. For example, assisting students in live programming or in-class pair-programming sessions was more difficult when they ran into individual errors with the coding module. It was also challenging to facilitate small group discussions. While it is easy to walk up to students to help them with technical difficulties during in-person instruction, we were forced to take advantage of Zoom's "breakout room" feature to assist these students. One student from CMM 262 taught in Winter 2021 commented on the interactive ChIP-sequencing analysis pipeline module: "\textit{This module would be one that would benefit from in-person instruction, because it was easy to fall behind during the coding segment. I didn't want to interrupt the class to slow down and I would have been more comfortable asking a TA or a neighbor in a physical classroom.}" Ultimately, it was difficult to assist students with conceptual or technical difficulties during lecture time. Often, these students would approach the teaching team during office hours to resolve any issues. 

Fortunately, there were some benefits to moving the course entirely online. Students, for example, could review recorded content during their own time. Two students from the Winter 2021 iteration of CMM 262 commented, "\textit{I think the recorded Zoom lectures help though, since I definitely needed to rewatch some parts}" and "\textit{…since the lectures are recorded, I am able to go back and go through it at my own pace, which is really helpful and appreciated!}" Another student commented on the same course, "\textit{For an online format, the course worked well when it came to being able to access the lecture recordings with captions since it can be hard to sit through an online lecture without them. I felt that the course was not adapted for longer lectures since I was experiencing Zoom fatigue and could not hold my attention for more than an hour (maybe note-taking-friendly formats or shorter, more frequent lectures may help).}" Properly deploying synchronous, practical bioinformatics classes requires instructors to consider how online mediums such as Zoom will impact students' learning experience. While we encountered logistical difficulties in interacting with students through Zoom breakout rooms or combating Zoom fatigue, the transition to online education, accelerated by the SARS-CoV-2 virus, underscored the possibility of making bioinformatics education more accessible to a broader audience. 

%%%%%%%%%%%%%%%%%%%%%%%%%%%%%%%%%%%%%%%%%%%%%%%%%%%%%%%%%%%%%%%%%%%%%%%%%%%%%%%%
\section{Conclusion}
%%%%%%%%%%%%%%%%%%%%%%%%%%%%%%%%%%%%%%%%%%%%%%%%%%%%%%%%%%%%%%%%%%%%%%%%%%%%%%%%

We are currently in a transition period in how we approach undergraduate Biology education from one that takes a surface-level approach in introducing bioinformatics analyses in one-off modules to one that integrates traditional computational courses into the canonical curriculum. While these changes will ultimately benefit the next generation of scientists in analyzing the large-scale biological datasets of the future, there is a need to address the knowledge gap for graduate students and other professional scientists of the present. While it is important to consider incorporating practical course modules into bioinformatics and balance the amount of material to include within bioinformatics classes, one of the largest considerations is the background of the students being taught.

Students wishing to learn bioinformatics later in their careers often come from various specialties spanning biological and biomedical sciences to the physical and computational sciences. Thus, one of the largest challenges in designing a comprehensive bioinformatics course is balancing these diverse backgrounds with designing course material that does not isolate students based on their knowledge gaps in theoretical biology, programming, and statistics. To teach bioinformatics to an academically diverse classroom, incorporating course materials that incorporate aspects of everybody's background help create a common ground for people to grow. Within SIOB 242C, CMM 262, and the BISB Bootcamp, showcasing computer science concepts of data types and looping in the context of analyzing genomic sequences proved successful in teaching biological and biomedical sciences students while cementing core instructional concepts and reducing the psychological barrier of stereotype threat. Slowly introducing theoretical biology in the context of interesting computational problems also successfully taught computational students without inflicting information overload. Balancing course content with students' learning abilities makes it possible to unify the classroom without leaving people behind. Additionally, introducing practical bioinformatics examples through student-paced live programming helps make bioinformatics accessible to new audiences and encourages an inclusive environment for all academic backgrounds.

%%%%%%%%%%%%%%%%%%%%%%%%%%%%%%%%%%%%%%%%%%%%%%%%%%%%%%%%%%%%%%%%%%%%%%%%%%%%%%%%
\section{Acknowledgments}
%%%%%%%%%%%%%%%%%%%%%%%%%%%%%%%%%%%%%%%%%%%%%%%%%%%%%%%%%%%%%%%%%%%%%%%%%%%%%%%%

I would like to thank Niema Moshiri, Clarence Mah, and Emma Farley for their thoughts and helpful feedback in writing this chapter. Additionally, in teaching introductory bioinformatics courses at UCSD, I learned from the students and from the other instructors I worked with in developing the course materials, lectures, and assignments. SIOB 242C, CMM 262, and the BISB Bootcamp courses would not have been successful without their dedicated support. 

Firstly, I am grateful to Alon Goren, Daniela "Dana" Nachmanson, Clarence Mah, Eric Kofman, Pratibha Jagannatha, and all of our guest instructors for being wonderful and flexible members of the teaching team for CMM 262 taught during the Winter Quarters of 2020 and 2021. Through teaching this class, I fine-tuned my knowledge of diverse bioinformatics pipelines and met many of the wonderful students in the Biomedical Sciences (BMS) graduate program. 

Next, I would like to thank Owen Chapman, Cameron Martino, Mike Cuoco, and Lauryn Bruce for their help in co-teaching the BISB-Biomedical Sciences (BMS) Joint Program Bootcamp in September 2020 and the BISB Bootcamp in September 2021. Planning a student-run Bootcamp in the limbo of the early COVID-19 pandemic on top of doing my thesis research was especially stressful, and I'm thankful to have had Owen and Cameron by my side to navigate the uncertainties of whether or not we would be able to teach a 50-person, primarily experimental class on how to code on the command line in a week. I would also like to thank the numerous guest student instructors from both Bootcamp sessions who took the time outside of research obligations to teach their peers various skills needed to survive both the personal and professional aspects of graduate school. This included Alexander Wenzel, Gibraan Rahman, Clarence Mah, Adam Officer, and George Armstrong from the BISB program, as well as Alex Tankka, Sara Elmsaouri, Danielle Schafer, Maya Gosztyla, Noorsher Ahmed, and Margaret Burns from the BMS program. Without these dedicated individuals, we would have never been able to cover as much material as we did, and thus you have my thanks! 

Finally, I would like to express my deepest gratitude to Terry Gaasterland for being an encouraging bioinformatics education mentor since I first took a variant of what is now SIOB 242C as an undergraduate student during Spring Quarter 2015. That class, SIO 190, sparked my interest in bioinformatics as a discipline and set the foundation for how I approach analyses today. I would also like to thank her for trusting me to assist in teaching SIOB 242C for the Spring Quarter 2017 course as a senior undergraduate student and then again during Spring Quarter 2019 as a first-year Ph.D. student. Without those experiences, I would not have been able to develop my passion for developing educational materials for students or experiment with incorporating new teaching methods into the classroom. It was also Terry's idea that I include this chapter within my thesis; for that, I will always be forever thankful that these reflections about teaching were not lost to time.

\begin{dissertationepilogue}
    %%%%%%%%%%%%%%%%%%%%%%%%%%%%%%%%%%%%%%%%%%%%%%%%%%%%%%%%%%%%%%%%%%%%%%%%%%%%
    \section{Conclusion}
    %%%%%%%%%%%%%%%%%%%%%%%%%%%%%%%%%%%%%%%%%%%%%%%%%%%%%%%%%%%%%%%%%%%%%%%%%%%%
    I started this work with a quote from Lewis Wolpert, "It is not birth, marriage, or death, but gastrulation that is the most important time of our lives." Indeed, gastrulation is a critical step in embryonic development. Gastrulation is when the primordial germ layers are specified, the embryonic axes manifest, and the embryo alters its morphology for the first time. Within chordates, the formation of the primary germinal layers-the ectoderm, mesoderm, and endoderm-requires meticulous control of individual cell and collective tissue behaviors with regard to space and time \cite{ghimire2021,balmer2016,winkley2020,solnica-krezel2012}. Thus, it is crucial to study the contents of a cell and the active regulatory factors at this stage to understand how defects in this machinery lead to congenital disabilities and disease. 
    
    One structure that emerges during gastrulation is the notochord, a rod-like, cartilaginous skeleton of mesodermal origin that defines chordates. The notochord serves as a signaling center for the embryonic midline and becomes an integral part of the vertebrate backbone as the nucleus pulposus of intervertebral discs \cite{solnica-krezel2012,balmer2016,debree2018,winkley2020,ghimire2021,stemple2004,stemple2005,choi2008,raj2008a,lawson2015}. Understanding notochord structure and function during gastrulation is essential to elucidate how perturbations to this machinery may lead to congenital vertebral defects. In this thesis dissertation, I demonstrated the importance of understanding the regulatory mechanisms driving gastrulation by studying the activity of enhancers and the contents of a cell during notogenesis. The ability to perform high-throughput experiments in the marine chordate \textit{Ciona intestinalis type A} or \textit{Ciona robusta} (\textit{Ciona}) contributed to the ability to screen the activity of thousands of enhancers to understand the contributions of transcription factor binding sites to function (Chapter \ref{chap:Diverse logics encode notochord enhancers}, Chapter \ref{chap:Proof of concept method to identify enhancers}). Likewise, \textit{Ciona} also granted us the unique ability to profile the transcriptomes of thousands of single cells in whole, gastrulating embryos to understand the contents of a cell across major tissues during cell type specification (Chapter \ref{chap:Ciona intestinalis gastrulation}).

    In Chapter \ref{chap:Diverse logics encode notochord enhancers}, we sought to understand the regulatory logic of notochord enhancers by taking advantage of the ability to perform high-throughput, massively-parallel reporter assays (MPRA) within \textit{Ciona}. Within the \textit{Ciona} genome, we identified 1,092 genomic regions, dubbed the ZEE library, containing a Zic binding site within 30 bp of an ETS binding site \cite{song2022}. Of the 90 ZEE elements, surprisingly, only nine drove notochord expression. One of the nine we identified, the \textit{Ciona} \textit{laminin alpha} enhancer, relied on grammatical constraints on Zic and ETS for functional activity. We also find similar clusters of Zic and ETS binding sites proximal to the mouse and human \textit{laminin alpha-1} gene with syntax similar to the \textit{Ciona laminin} enhancer \cite{song2022}. Within this chapter, we also highlight the importance of testing the sufficiency of TFBSs to investigate if we fully understand the regulatory logic of an enhancer. Through a randomization study, we reveal within the previously identified BraS enhancer that Zic and ETS binding sites are insufficient for notochord activity. Furthermore, we also find that FoxA and Bra sites are also necessary for notochord expression with BraS and that the combination of Zic, ETS, FoxA, and Bra binding sites may be a common logic regulating Bra expression \cite{song2022}. Our findings in Chapter \ref{chap:Diverse logics encode notochord enhancers} illustrate a common problem in mining genomic data for patterns, especially in mining genomes for functional enhancers based on the presence of TFBSs. We demonstrate that the presence of binding sites alone does not correlate to enhancer activity. To understand how enhancers regulate gene expression, we need to understand the number and types of TFBSs within an enhancer and the dependency between these sites, such as TFBS syntax and affinity \cite{jindal2021}. Overall, our findings illustrate the importance of enhancer grammar within developmental enhancers and hint at the conserved role of grammar and logic across chordates. 

    In Chapter \ref{chap:Proof of concept method to identify enhancers}, we continue upon the framework of Chapter \ref{chap:Diverse logics encode notochord enhancers} to understand the regulatory logic of notochord enhancers consisting of Zic, ETS, Bra, and FoxA binding sites. By virtue of a new \textit{Ciona} genome reference sequence release in 2019 \cite{satou2019}, we found a total of 4,344 genomic elements containing Zic and ETS binding sites with flexible constraints of the position of these sites within a 100 bp window. This library is otherwise known as the KYN library. After testing these KYN elements in an MPRA in whole \textit{Ciona} embryos, we found that only 15.4\% of these sites were active and dependent on Zic and ETS binding sites. Reviewing several candidate enhancers, we find they are proximal to multiple genes implicated in nervous system disorders and skeletal and bone-related disorders. Ultimately, further study of this enhancer library through imaging studies and TFBS ablation experiments is needed to ascertain the dependency of binding sites in determining functionality. To finish this chapter, we introduced a proof-of-concept Python package, \textbf{E}ntire \textbf{G}enome se\textbf{A}rches for \textbf{G}rammars of \textbf{E}nhancers (EnGAGE), that was developed to aid in efforts to understand the connection between genomic sequence and regulatory activity. This work represents the beginnings of a new paradigm to understand enhancers through elucidating how the organization of collections of TFBSs contributes to functional activity. 

    Finally, we move beyond genomic sequence to the contents of a cell in Chapter \ref{chap:Ciona intestinalis gastrulation}, where we develop a high-throughput, dense transcriptional atlas of \textit{Ciona} gastrulation. Just as the specific activity of enhancers is essential for successful development, the contents of a cell also dictate the formation of key cell types, such as the epidermis, endoderm, mesenchyme, heart, muscle, germ cells, notochord, and nervous system. In this study, we develop \textit{Ciona} embryos to the time points dictating gastrulation-the 4.5 hours post fertilization (hpf), 5.5 hpf, and 6.5 hpf stages representing the early gastrula or 110-cell stage, late gastrula, and early neurula stages of development \cite{satoh2014}. Once developed, the embryos are rapidly disassociated and processed for single-cell RNA sequencing. In total, we were able to profile 356,671 cells, allowing us to identify major tissues undergoing organogenesis and rare cell-type populations, such as the developing heart and germ cells. We also validate our map with fluorescent \textit{in situ} hybridization (FISH) imaging studies, visualizing canonical marker genes and novel marker genes within late gastrula \textit{Ciona} embryos. By providing a higher resolution single-cell atlas just spanning gastrulation, we anticipate that other groups can use the map generated in this study to identify conserved canonical and novel cell differentiation markers. Additionally, this resource will provide insight into the cell fate mechanisms governing organ formation during \textit{Ciona} gastrulation. 

    In the first three chapters of this work, I interrogate the regulatory players driving notogenesis from a genomic perspective through understanding the impact of binding site dependencies within enhancers (Chapter \ref{chap:Diverse logics encode notochord enhancers}, Chapter \ref{chap:Proof of concept method to identify enhancers}) and from a cellular perspective through cataloging the contents of major cell types by creating a transcriptional atlas of the developing \textit{Ciona} gastrula (Chapter \ref{chap:Ciona intestinalis gastrulation}). I also demonstrate that we cannot rely on bioinformatic identification of putative enhancers based on TFBS presence alone. In addition to elucidating the mechanisms driving notochord enhancer activity and the transcriptional landscape driving organogenesis, I also make the argument for studying enhancer grammar. To identify developmental enhancers accurately from genomic sequences, we need to understand the number and types of TFBSs present within a sequence and the dependency between these sites regarding syntax and binding affinity. Additionally, we need to understand the cellular context and transcriptional landscape in which these sequences are active. By studying these two elements in tandem, we can further understand how we transform from a single cell to a multicellular organism. 

    Finally, in the last chapter of this work, I demonstrate the importance of teaching bioinformatics and the strategies for managing an academically diverse classroom. The work presented in the first three chapters of this thesis dissertation required an understanding of programming, data visualization, molecular and developmental biology, and statistics to comment on enhancer grammar and cell type specification (Chapter \ref{chap:Diverse logics encode notochord enhancers}, Chapter \ref{chap:Proof of concept method to identify enhancers}, Chapter \ref{chap:Ciona intestinalis gastrulation}). As molecular biology steps into the world of big data to understand regulatory genomics, scientists must pick up bioinformatics skills. In Chapter \ref{chap:Bioinformatics education}, I share my experiences teaching bioinformatics curricula at the university level through the SIOB 242C, CMM 262, and BISB Bootcamp courses offered at the University of California, San Diego. One of the most considerable challenges I encountered in developing a comprehensive bioinformatics course was balancing the diverse academic backgrounds of students with creating experiences that do not isolate students based on their knowledge gaps in molecular biology and computer programming. To create a community environment in the classroom, I discuss my strategies in slowly introducing computational and biological concepts to reduce information overload and combat stereotype threat. Additionally, I discuss the benefits of introducing practical bioinformatics examples through student-paced, classroom-wide live programming sessions. Through this work, I want to enforce that learning bioinformatics can be made accessible through proper course design and empathetic instruction.

    %%%%%%%%%%%%%%%%%%%%%%%%%%%%%%%%%%%%%%%%%%%%%%%%%%%%%%%%%%%%%%%%%%%%%%%%%%%%
    \section{Limitations and Future Directions}
    %%%%%%%%%%%%%%%%%%%%%%%%%%%%%%%%%%%%%%%%%%%%%%%%%%%%%%%%%%%%%%%%%%%%%%%%%%%%
    Though this work represents essential steps forward in understanding the mechanisms behind enhancer grammar and cell type specification during gastrulation, there are still many avenues of study to continue down and important limitations to keep in mind.

    In Chapter \ref{chap:Diverse logics encode notochord enhancers} and Chapter \ref{chap:Proof of concept method to identify enhancers}, there are limitations regarding identifying functional enhancers and the ability to translate grammatical principles across species. For example, within Chapter \ref{chap:Diverse logics encode notochord enhancers}, we screened 90 ZEE elements for functionality; however, only 10\% were active in the notochord. Additionally, in Chapter \ref{chap:Proof of concept method to identify enhancers}, we screen for 4,344 KYN elements for functionality; however, only 15.4\% are active and reliant on Zic and ETS. While we anticipate that finding more notochord enhancers regulated by Zic, ETS, and possibly Bra and FoxA could better inform our understanding of the notochord enhancer grammar, finding these regions is highly limited. Combining assays of genomic regions with synthetic and random enhancer screens is thus needed to gain enough data to determine grammatical rules. With regards to our findings of possible conserved enhancer logic and grammar across chordates, we did not test the mouse \textit{laminin alpha-1} enhancer for activity in mouse for the study presented in Chapter \ref{chap:Diverse logics encode notochord enhancers}. We also did not functionally interrogate the importance of the 12 bp spacing within this enhancer in the context of \textit{Ciona} or mouse. Conducting these additional studies would deepen our understanding of the conservation of grammar across chordates. On the other hand, for the Zic, ETS, Bra, and FoxA logic found within \textit{Brachyury} enhancers in Chapter \ref{chap:Diverse logics encode notochord enhancers}, further manipulations of these TFBSs in the context of mouse and zebrafish \textit{Brachyury}/\textit{T}/\textit{TBXT} enhancers are required to determine if the conservation of logic is essential for the regulation of \textit{Brachyury}. Similar interrogations into the importance of binding sites in the active enhancers identified in Chapter \ref{chap:Proof of concept method to identify enhancers} are necessary to evaluate the components of the enhancer required for activity. 

    In Chapter \ref{chap:Ciona intestinalis gastrulation}, I present a high-resolution transcriptional atlas encompassing \textit{Ciona} gastrulation. Despite our success in identifying key cell types and sub-clusters representing their original cell-type lineages in \textit{Ciona} (e.g., A-line, B-line, a-line, and b-line), we did not evaluate the cell lineage specification pathways or pseudotime trajectories in the formation of these cell types. There is increasing interest in understanding the transitionary states involved in cell type specification. Within our dataset, one avenue for future study could be the initial formation of the notochord from mesenchymal tissue or the formation of initial neural subtypes from the A-line, a-line, and b-line cell lineages of the \textit{Ciona} embryo. Uncovering the markers delineating particular states in these trajectories, especially in a high-resolution single-cell atlas, may uncover additional essential marker genes involved in initial organ formation. In addition to studying cell type specification patterns through constructing pseudotime trajectories, another avenue for future work includes annotation of genes present in the \textit{Ciona} gastrula. Within our high-resolution single-cell atlas, we were able to identify not only canonical markers within cell types but also many novel markers lacking clear definitions on Aniseed besides sequence homology to vertebrate homologs. A straightforward avenue for future work is visualizing these novel markers through imaging experiments to validate their expression in the cell types identified in our single-cell atlas and define these genes for the larger \textit{Ciona} community. These studies would also confirm or deny the value of sequence homology in determining the true activity of novel markers.
    
\end{dissertationepilogue}

% Command that provides dummy text to showcase formatting:
% \Blinddocument

%%%%%%%%%%%%%%%%%%%%%%%%%%%%%%%%%%%%%%%%%%%%%%%%%%%%%%%%%%%%%%%%%%%%%%%%%%%%%%%%
% Appendix of the Dissertation
%%%%%%%%%%%%%%%%%%%%%%%%%%%%%%%%%%%%%%%%%%%%%%%%%%%%%%%%%%%%%%%%%%%%%%%%%%%%%%%%
\appendix

%%%%%%%%%%%%%%%%%%%%%%%%%%%%%%%%%%%%%%%%%%%%%%%%%%%%%%%%%%%%%%%%%%%%%%%%%%%%%%%%
% End of the Dissertation
%%%%%%%%%%%%%%%%%%%%%%%%%%%%%%%%%%%%%%%%%%%%%%%%%%%%%%%%%%%%%%%%%%%%%%%%%%%%%%%%
\backmatter{}
\bibliographystyle{unsrtnat} 
\bibliography{bibliography}

\end{document}
